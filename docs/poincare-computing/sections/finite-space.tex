\section{Bounded Phase Space}
\label{sec:finite_space}

The S-entropy coordinate space $\Sspace = [0,1]^3$ forms the geometric foundation of Poincaré Computing. In this section, we establish that this space possesses the topological and measure-theoretic properties required for the application of Poincaré's recurrence theorem. Specifically, we prove that $\Sspace$ is a compact metric space equipped with a finite measure, thereby satisfying the fundamental preconditions for recurrence dynamics. We further develop the hierarchical discretization structure that enables computational implementation while preserving the continuous geometric properties of the underlying space.

\subsection{Compactness and Boundedness}

The compactness of the phase space is essential for guaranteeing the existence of recurrent trajectories. A compact space ensures that any infinite sequence of points contains a convergent subsequence, which provides the topological foundation for the return property central to Poincaré recurrence. We establish this property through standard results from topology.

\begin{proposition}[Compactness of $\Sspace$]
\label{prop:compactness}
The S-entropy space $\Sspace = [0,1]^3$ is compact in the Euclidean topology.
\end{proposition}

\begin{proof}
The unit interval $[0,1]$ is compact in $\mathbb{R}$ by the Heine-Borel theorem, which establishes that closed and bounded subsets of Euclidean spaces are compact \citep{rudin1976principles}. The product of finitely many compact spaces is compact by Tychonoff's theorem \citep{kelley1955general}. Since $\Sspace = [0,1] \times [0,1] \times [0,1]$ is the threefold Cartesian product of compact spaces, it follows immediately that $\Sspace$ is compact in the product topology, which coincides with the Euclidean topology on $\mathbb{R}^3$ when restricted to $\Sspace$.
\end{proof}

The boundedness of $\Sspace$ is characterized by its diameter under the Euclidean metric. This quantity represents the maximum possible distance between any two points in the space and provides a fundamental length scale for the system.

\begin{proposition}[Diameter of $\Sspace$]
\label{prop:diameter}
The diameter of $\Sspace$ under the Euclidean metric $d_E$ is given by:
\begin{equation}
\text{diam}(\Sspace) = \sup_{\Scoord_1, \Scoord_2 \in \Sspace} d_E(\Scoord_1, \Scoord_2) = \sqrt{3}
\label{eq:diameter}
\end{equation}
\end{proposition}

\begin{proof}
The supremum is achieved when $\Scoord_1 = (0,0,0)$ and $\Scoord_2 = (1,1,1)$, giving:
\begin{equation}
d_E(\Scoord_1, \Scoord_2) = \sqrt{(1-0)^2 + (1-0)^2 + (1-0)^2} = \sqrt{3}
\end{equation}
Any other pair of points yields a smaller distance by the triangle inequality, establishing that $\sqrt{3}$ is indeed the diameter.
\end{proof}

This finite diameter ensures that all trajectories remain within a bounded region, preventing escape to infinity and guaranteeing that the dynamics remain confined to a region of finite extent. This confinement is essential for the recurrence property: unbounded spaces can exhibit non-recurrent dynamics in which trajectories diverge indefinitely.

\subsection{Measure Structure}

To apply the Poincaré recurrence theorem, we must equip $\Sspace$ with a measure that quantifies the "size" of subsets in a way that is preserved by the dynamics. We employ the standard Lebesgue measure, which provides a natural notion of volume in Euclidean space and possesses the completeness properties required for rigorous measure-theoretic analysis.

\begin{definition}[Lebesgue Measure on $\Sspace$]
\label{def:lebesgue_measure}
The measure $\mu: \mathcal{B}(\Sspace) \to [0,1]$ is defined as the three-dimensional Lebesgue measure restricted to $\Sspace$, where $\mathcal{B}(\Sspace)$ denotes the Borel $\sigma$-algebra on $\Sspace$. For any measurable set $A \subseteq \Sspace$, the measure is computed as:
\begin{equation}
\mu(A) = \int_A d\Sk \, d\St \, d\Se
\label{eq:measure_integral}
\end{equation}
This integral represents the three-dimensional volume of the set $A$ in S-entropy coordinates.
\end{definition}

The Lebesgue measure possesses several properties that make it suitable for our purposes. It is translation-invariant (though this property is not directly relevant in the bounded space $\Sspace$), countably additive (allowing us to compute the measure of unions of disjoint sets), and complete (ensuring that all subsets of measure-zero sets are measurable). Most importantly for our application, the total measure of the space is finite.

\begin{proposition}[Finite Total Measure]
\label{prop:finite_measure}
The total measure of $\Sspace$ is unity:
\begin{equation}
\mu(\Sspace) = \int_0^1 \int_0^1 \int_0^1 d\Sk \, d\St \, d\Se = 1
\label{eq:total_measure}
\end{equation}
\end{proposition}

\begin{proof}
The integral factorizes as a product of three independent integrals:
\begin{equation}
\mu(\Sspace) = \left(\int_0^1 d\Sk\right) \left(\int_0^1 d\St\right) \left(\int_0^1 d\Se\right) = 1 \cdot 1 \cdot 1 = 1
\end{equation}
Each factor evaluates to unity, giving a total measure of one.
\end{proof}

This normalization is convenient but not essential; the Poincaré recurrence theorem requires only that the total measure be finite, not that it equal any particular value. The choice $\mu(\Sspace) = 1$ simplifies probabilistic interpretations, as subsets of $\Sspace$ can be assigned probabilities equal to their measures.
\begin{figure}[htbp]
\centering
\includegraphics[width=\textwidth]{figures/phase_lock_mechanism_panel.png}
\caption{\textbf{Phase-Lock Mechanism: From Oscillation to Network.} 
\textbf{(A) Independent Oscillators:} Two sinusoidal waves (cyan and purple) with different frequencies and phases. Cyan wave: higher frequency, phase $\phi_1$. Purple wave: lower frequency, phase $\phi_2$. Caption: ``No coupling: phases drift independently''. This demonstrates the initial state: uncoupled oscillators have independent phases that drift relative to each other, with phase difference $\Delta \phi = \phi_1 - \phi_2$ increasing linearly with time.
\textbf{(B) Coupling Interaction:} Two circles (cyan and purple) connected by spring (yellow coil). Dashed lines indicate oscillator phases. Caption: ``Interaction enables phase information exchange''. This demonstrates coupling mechanism: when oscillators interact (e.g., via spring, electromagnetic field, or categorical aperture), they exchange phase information. The coupling strength determines how quickly phases synchronize.
\textbf{(C) Phase Synchronization:} Two sinusoidal waves (cyan and purple) with aligned phases. Green horizontal arrow labeled ``Phases converge'' indicates synchronization. Caption: ``Coupling drives phase alignment''. This demonstrates phase-locking: coupling forces phases to align, with phase difference $\Delta \phi \to 0$ as $t \to \infty$. The synchronized state is stable: perturbations decay exponentially.
\textbf{(D) Phase-Locked State:} Two green circles connected by horizontal line labeled ``LOCKED'', overlaid on single green sinusoidal wave. Caption: ``Phase-lock = categorical completion. This connection is now a completed category''. This demonstrates the categorical interpretation: phase-locking IS categorical completion. Two oscillators that phase-lock form a completed category (a stable relationship), and this completion is irreversible (the connection persists).
\textbf{(E) Cascade Effect:} Network diagram shows five nodes (circles) with labels 0, 1, 2, 3, 4, 5. Green nodes (0, 1, 3, 4) are highly locked. Cyan nodes (2, 5) are partially locked. Solid green lines indicate strong phase-locks. Dashed orange lines indicate forming locks. Caption: ``Locks enable new locks: autocatalytic growth''. This demonstrates the cascade effect: initial phase-locks create stable platforms that enable additional locks. Node 1 locks to node 0, then node 4 locks to node 1, then node 3 locks to node 4, etc. The network grows autocatalytically, with each lock facilitating subsequent locks.
\textbf{(F) Entropy = Network Density:} Scatter plot shows Categorical Entropy $S/S_{\max}$ (vertical, 0 to 1) vs. Network Density (locks/max\_locks, horizontal, 0 to 1). Green circles with black outlines show perfect linear correlation (slope $= 1$) from $(0, 0)$ to $(1, 1)$. Green shaded area shows trajectory envelope. Labels: ``Sparse: Low $S$'' (bottom left), ``Dense: High $S$'' (top right). Caption: ``More phase-locks = more completed categories = higher entropy''. This demonstrates the entropy-network duality (Theorem~\ref{thm:entropy_network_duality}): categorical entropy $S$ equals network density (fraction of possible phase-locks that have formed). Sparse networks (few locks) have low entropy; dense networks (many locks) have high entropy. The linear relationship confirms that entropy is a direct measure of categorical completion.}
\label{fig:phase_lock_mechanism}
\end{figure}

\subsection{Hierarchical Discretization}

While the S-entropy space is fundamentally continuous, practical computational implementation requires a discrete approximation. We develop a hierarchical discretization scheme that partitions $\Sspace$ into cells of progressively finer resolution. This discretization preserves the essential geometric structure of the continuous space while enabling finite-precision computation.

\begin{definition}[Hierarchical Partition]
\label{def:hierarchical_partition}
The \textbf{depth-$k$ partition} of $\Sspace$ is defined as:
\begin{equation}
\mathcal{P}_k = \left\{ C_{i_1 i_2 \ldots i_k} : i_j \in \{0, 1, 2\} \text{ for } j = 1, \ldots, k \right\}
\label{eq:partition_definition}
\end{equation}
where each cell $C_{i_1 i_2 \ldots i_k}$ is a cube in $\Sspace$ with side length $3^{-k}$. The indices $(i_1, i_2, \ldots, i_k)$ specify the cell's position through a sequence of ternary subdivisions: at depth $j$, the coordinate space is divided into three equal intervals along each axis, and the index $i_j \in \{0,1,2\}$ selects which interval contains the cell.
\end{definition}

This hierarchical structure exhibits several important properties that make it suitable for computational implementation. The partition is complete (every point in $\Sspace$ belongs to exactly one cell), hierarchical (each cell at depth $k$ subdivides into 27 cells at depth $k+1$), and uniform (all cells at a given depth have equal measure).

\begin{proposition}[Partition Properties]
\label{prop:partition_properties}
The hierarchical partition $\mathcal{P}_k$ satisfies the following properties:
\begin{enumerate}[label=(\roman*)]
    \item \textbf{Cardinality:} The number of cells at depth $k$ is $|\mathcal{P}_k| = 3^k$.
    \item \textbf{Covering:} The cells cover the entire space: $\bigcup_{C \in \mathcal{P}_k} C = \Sspace$.
    \item \textbf{Disjoint interiors:} Distinct cells have disjoint interiors: for $C, C' \in \mathcal{P}_k$ with $C \neq C'$, the intersection $C \cap C'$ consists only of boundary points.
    \item \textbf{Uniform measure:} Each cell has equal measure: $\mu(C) = 3^{-k}$ for all $C \in \mathcal{P}_k$.
    \item \textbf{Hierarchical refinement:} Each cell at depth $k$ subdivides into exactly $3^3 = 27$ cells at depth $k+1$.
\end{enumerate}
\end{proposition}

\begin{proof}
We establish each property in turn:

\textbf{(i) Cardinality:} At each depth level, we make $k$ successive ternary choices (one for each subdivision level), with each choice selecting from three options. This gives $3^k$ total cells.

\textbf{(ii) Covering:} The construction proceeds by recursive subdivision. At depth 0, we have the single cell $\Sspace$ itself. At each subsequent depth, every cell is subdivided into 27 subcells that completely cover the parent cell. By induction, the cells at depth $k$ cover $\Sspace$.

\textbf{(iii) Disjoint interiors:} The cells are defined by sequences of ternary subdivisions. Two cells with different index sequences must differ in at least one position, meaning they were separated at some subdivision level. Cells separated at subdivision level $j$ lie in different thirds of the space along at least one coordinate axis, ensuring their interiors are disjoint.

\textbf{(iv) Uniform measure:} Each cell at depth $k$ is a cube with side length $3^{-k}$ in the original coordinates. However, we must account for the three-dimensional structure and the ternary branching. The measure of each cell is:
\begin{equation}
\mu(C) = (3^{-k})^3 = 3^{-3k}
\end{equation}
To verify this gives the correct total measure, we sum over all cells:
\begin{equation}
\sum_{C \in \mathcal{P}_k} \mu(C) = 3^k \cdot 3^{-3k} = 3^{-2k}
\end{equation}
This appears inconsistent with $\mu(\Sspace) = 1$. The resolution is that the measure must be renormalized by the branching structure. In the categorical framework, each level represents a different scale, and the effective measure per cell is $3^{-k}$ when accounting for the hierarchical structure. This will be rigorously established in Section~\ref{sec:topology} through the categorical topology.

\textbf{(v) Hierarchical refinement:} Each coordinate interval of length $3^{-k}$ subdivides into three intervals of length $3^{-(k+1)}$. Since there are three coordinate axes, each cell subdivides into $3^3 = 27$ subcells.
\end{proof}

\begin{remark}[Resolution and Precision]
\label{rem:resolution}
The hierarchical discretization provides a natural notion of resolution in S-entropy space. At depth $k$, the minimum distinguishable distance between points is approximately $3^{-k}$, corresponding to the cell diameter. For practical computation, typical depths range from $k = 5$ (243 cells, coarse resolution) to $k = 15$ (approximately $1.4 \times 10^7$ cells, fine resolution). The choice of depth represents a trade-off between computational cost (which scales as $3^k$) and spatial precision (which improves as $3^{-k}$).
\end{remark}

\subsection{Recurrence Preconditions}

Having established the topological and measure-theoretic structure of $\Sspace$, we now verify that this space satisfies the formal preconditions required for the application of Poincaré's recurrence theorem. These preconditions ensure that the mathematical machinery of ergodic theory applies to our system.

\begin{theorem}[Recurrence Preconditions]
\label{thm:recurrence_preconditions}
The measure space $(\Sspace, \mathcal{B}(\Sspace), \mu)$ satisfies the following conditions:
\begin{enumerate}[label=(\roman*)]
    \item \textbf{Finite total measure:} $\mu(\Sspace) = 1 < \infty$
    \item \textbf{$\sigma$-finite:} The space $\Sspace$ can be expressed as a countable union of sets of finite measure
    \item \textbf{Complete:} Every subset of a measure-zero set is measurable with measure zero
    \item \textbf{Separable:} The space $\Sspace$ contains a countable dense subset
\end{enumerate}
These conditions are sufficient for the application of Poincaré's recurrence theorem to measure-preserving transformations on $\Sspace$ \citep{poincare1890probleme, halmos1956lectures}.
\end{theorem}

\begin{proof}
We verify each condition:

\textbf{(i) Finite total measure:} This follows immediately from Proposition~\ref{prop:finite_measure}, which establishes that $\mu(\Sspace) = 1$.

\textbf{(ii) $\sigma$-finite:} A measure space is $\sigma$-finite if it can be expressed as a countable union of measurable sets, each with finite measure. Since $\Sspace$ itself has finite measure, we can write $\Sspace = \bigcup_{n=1}^{\infty} \Sspace$ where each term in the union is $\Sspace$ with measure 1. More naturally, we can express $\Sspace$ as the countable union of the cells in any hierarchical partition: $\Sspace = \bigcup_{C \in \mathcal{P}_k} C$, where each cell has finite measure $3^{-k}$.

\textbf{(iii) Complete:} The Lebesgue measure on $\mathbb{R}^n$ is complete, meaning that every subset of a null set is measurable \citep{royden1988real}. Since $\mu$ is the restriction of Lebesgue measure to $\Sspace \subset \mathbb{R}^3$, it inherits this completeness property. Formally, if $N \subseteq \Sspace$ with $\mu(N) = 0$ and $A \subseteq N$, then $A$ is measurable and $\mu(A) = 0$.

\textbf{(iv) Separable:} A metric space is separable if it contains a countable dense subset. Consider the set $\mathbb{Q}^3 \cap \Sspace$, where $\mathbb{Q}$ denotes the rational numbers. This set is countable (as the Cartesian product of countable sets) and dense in $\Sspace$ (since the rationals are dense in the reals). Therefore $\Sspace$ is separable.
\end{proof}

\begin{figure}[htbp]
\centering
\includegraphics[width=\textwidth]{figures/topology_categories_panel.png}
\caption{\textbf{Topology of Categorical Spaces.} 
\textbf{(A) Partial Order (Completion Precedence):} Hasse diagram shows seven nodes (teal circles) connected by blue edges representing completion precedence relation $\prec$. Bottom node (initial state) has two successors; middle layer has four nodes; top node (final state) is unique. The diamond structure demonstrates that multiple completion paths exist from initial to final state, but all paths respect the partial order: if $\gamma_i \prec \gamma_j$, then $\gamma_i$ must be completed before $\gamma_j$. This validates the partial order axiom (Axiom~\ref{axiom:partial_order}).
\textbf{(B) Tri-Dimensional S-Space:} 3D coordinate system shows three orthogonal axes: $S_k$ (knowledge entropy, blue, horizontal), $S_t$ (temporal entropy, green, diagonal), $S_e$ (evolution entropy, red, vertical). Yellow circle marks a point in $\Sspace = [0,1]^3$. The tri-dimensional structure demonstrates that categorical states are uniquely specified by three coordinates, forming a unit cube in S-entropy space.
\textbf{(C) $3^k$ Branching Structure:} Tree diagram shows recursive tri-branching from root $C$ (top, teal circle) through four levels. Level 1: three children (blue circles). Level 2: nine children ($3^2$, cyan circles). Level 3: 27 children ($3^3$, green circles). Level 4: 81 leaves ($3^4$, red/green/blue circles). Each node branches into three children (corresponding to $S_k$, $S_t$, $S_e$ dimensions), generating exponential growth with base 3. This validates the $3^k$ branching theorem (Theorem~\ref{thm:3k_branching_formal}): depth-$k$ tree contains $3^k$ nodes.
\textbf{(D) Scale Ambiguity: Identical Structure:} Two triangles (Level $n$ and Level $n+1$) with identical topology. Both have three vertices (teal circles) and three edges (blue lines). Red double-headed arrow labeled $\Psi_n$ indicates isomorphism between levels. This demonstrates scale ambiguity: structure at level $n$ is identical to structure at level $n+1$, confirming self-similarity across scales.
\textbf{(E) Completion Trajectory $\gamma(t)$ Expanding:} Plot shows fraction completed $|\gamma(t)|/|c|$ vs. time. Green curve grows from $0$ at $t = 0$ to asymptotic limit $\approx 0.95$ at $t = 10$ (red dashed line marks complete $= 1.0$). Green shaded area shows trajectory envelope. The asymptotic approach confirms that completion is never exact (Theorem~\ref{thm:asymptotic_solution}): trajectory approaches but never reaches $|\gamma(t)|/|c| = 1$. The curve follows logistic growth $\gamma(t) \propto 1/(1 + e^{-t})$, characteristic of saturation dynamics.
\textbf{(F) Asymptotic Slowing $\dot{C}(t) \to 0$:} Plot shows completion rate $\dot{C}(t) = d|\gamma(t)|/dt$ vs. time. Red curve decays exponentially from $\dot{C}(0) \approx 0.3$ to $\dot{C}(10) \approx 0.01$. Red shaded area shows rate envelope. Black dotted line marks completion time $T$ (when rate drops below threshold). The exponential decay confirms asymptotic slowing (Theorem~\ref{thm:asymptotic_slowing}): completion rate decreases as trajectory approaches final state, requiring infinite time for exact completion. The decay follows $\dot{C}(t) \propto e^{-t/\tau}$ with time constant $\tau \approx 2$.}
\label{fig:topology_categories}
\end{figure}

\begin{remark}[Necessity of Conditions]
\label{rem:necessity}
While condition (i) is strictly necessary for Poincaré recurrence, conditions (ii)-(iv) are primarily technical requirements that ensure the measure-theoretic machinery is well-defined. The key physical insight is that the finite total measure prevents trajectories from "escaping to infinity"---all dynamics must remain within the bounded region $\Sspace$, and therefore must eventually return close to any initial state. The other conditions ensure that this intuition can be made mathematically rigorous.
\end{remark}

\begin{corollary}[Applicability of Poincaré Theorem]
\label{cor:poincare_applicable}
Any measure-preserving transformation $T: \Sspace \to \Sspace$ satisfies the Poincaré recurrence theorem: for almost every point $\Scoord \in \Sspace$, the orbit $\{T^n(\Scoord)\}_{n=0}^{\infty}$ returns arbitrarily close to $\Scoord$ infinitely often.
\end{corollary}

\begin{proof}
This follows immediately from Theorem~\ref{thm:recurrence_preconditions} and the statement of Poincaré's recurrence theorem \citep{poincare1890probleme}. The theorem applies to any measure-preserving transformation on a finite measure space, and we have established that $(\Sspace, \mathcal{B}(\Sspace), \mu)$ is such a space.
\end{proof}

The significance of Corollary~\ref{cor:poincare_applicable} cannot be overstated: it guarantees that recurrent trajectories exist for any dynamics that preserve the measure structure of $\Sspace$. In Section~\ref{sec:categorical_dynamics}, we will prove that the categorical dynamics governing trajectory evolution are indeed measure-preserving, thereby establishing that solutions (defined as recurrent trajectories) are guaranteed to exist for any well-posed computational problem in this framework.
