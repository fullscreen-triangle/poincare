\section{Hardware Grounding: Virtual Instrumentation}
\label{sec:virtual_instrument}

The S-entropy coordinate space developed in Section~\ref{sec:finite_space} is not merely an abstract mathematical construction but is instantiated through physical measurements from hardware oscillators present in every computing system. This section establishes the crucial mapping from physical timing measurements to categorical states in $\Sspace$, demonstrating that Poincaré Computing is grounded in measurable physical quantities rather than symbolic abstractions.

The key insight is that modern computing hardware contains multiple oscillatory processes---CPU clocks, memory buses, power supply ripples---whose timing relationships encode information about the computational state. By measuring the phase differences between these oscillators with high precision, we can construct coordinates in S-entropy space that reflect the actual physical state of the computing system. This approach transforms the computer from a discrete symbol manipulator into a continuous dynamical system whose trajectory through phase space constitutes the computation itself.

We develop three coordinate mapping functions $\phi_k$, $\phi_t$, and $\phi_e$ that transform timing measurements into the knowledge, temporal, and evolution entropy coordinates respectively. These mappings are deterministic, ensuring that identical physical states produce identical categorical states, while also being bounded to guarantee that all measurements map to valid points in $\Sspace = [0,1]^3$. We establish a fundamental result called the \textbf{spectrometer-state identity}, which proves that the measurement apparatus and the measured state are encoded by the same point in S-entropy space---a property with profound implications for the nature of observation in this computational framework.

\begin{figure}[htbp]
\centering
\includegraphics[width=\textwidth]{figures/instrument_suite_panel.png}
\caption{\textbf{Exotic Instrument Suite: Element Identification Through Measurement.} 
\textbf{Shell Resonator:} Resonance frequency decreases exponentially with shell number $n$, following $f_n \propto 1/n^2$ (analogous to hydrogen energy levels). Shell $n = 1$ shows strongest resonance ($\approx 1.0$ GHz), decaying to $\approx 0.05$ GHz by $n = 7$. This instrument measures the principal quantum number analog in categorical space. 
\textbf{Angular Analyzer (Subshell Capacity):} Pie chart shows relative populations of subshells $s$, $p$, $d$, $f$ (red, teal, blue, cyan sectors). The $f$-subshell dominates ($\approx 50\%$), followed by $d$ ($\approx 30\%$), $p$ ($\approx 15\%$), and $s$ ($\approx 5\%$). Capacity follows $2\ell + 1$ rule: $s$ (2 states), $p$ (6 states), $d$ (10 states), $f$ (14 states). This instrument measures the azimuthal quantum number $\ell$ analog. 
\textbf{Chirality Discriminator (Spin State):} Circle diagram with vertical axis showing spin-up ($+1/2$, red arrow) and spin-down ($-1/2$, cyan arrow) states. Binary discrimination implements magnetic quantum number $m_s \in \{-1/2, +1/2\}$ analog. This instrument measures the spin quantum number. 
\textbf{Spectral Analyzer (H Balmer Series):} Emission lines at characteristic wavelengths: $410$ nm (violet, $n = 6 \to 2$), $434$ nm (blue, $n = 5 \to 2$), $486$ nm (cyan, $n = 4 \to 2$), $656$ nm (red, $n = 3 \to 2$). Line heights represent transition intensities. This instrument identifies categorical states by their spectral signatures, analogous to atomic spectroscopy. 
\textbf{Ionization Probe (Period 2):} Ionization energy IE increases monotonically across period 2 elements: Li ($\approx 5.4$ eV), Be ($\approx 9.3$ eV), B ($\approx 8.3$ eV), C ($\approx 11.3$ eV), N ($\approx 14.5$ eV), O ($\approx 13.6$ eV), F ($\approx 17.4$ eV), Ne ($\approx 21.6$ eV, yellow bar). The trend reflects increasing nuclear charge $Z$ and decreasing atomic radius. This instrument measures the energy required to complete a categorical state. 
\textbf{Atomic Radius Gauge:} Colored circles show decreasing atomic radius across period 2: Li (purple, largest) $\to$ Ne (orange, smallest). Radius decreases due to increasing effective nuclear charge $Z_{\text{eff}}$ pulling electrons closer. This instrument measures the spatial extent of categorical states. 
\textbf{Measurement Workflow (Bottom):} Six-stage pipeline implements partition coordinate measurement: (1) Shell Resonator measures $n$, (2) Angular Analyzer measures $\ell$, (3) Orientation Mapper measures $m_\ell$, (4) Chirality Discriminator measures $m_s$, (5) Exclusion Detector enforces Pauli principle, (6) Energy Profiler applies Aufbau ordering. Each instrument measures one coordinate, collectively determining the element (categorical state identity). The workflow demonstrates that measurement IS computation: identifying the element requires measuring all partition coordinates, and the measurement process itself constitutes the identification.}
\label{fig:exotic_instrument_suite}
\end{figure}

\subsection{Oscillator Sources}

Modern computing hardware, despite being designed for discrete digital logic, contains numerous oscillatory processes operating at characteristic frequencies spanning many orders of magnitude. These oscillators are not incidental artifacts but are essential to the hardware's operation, providing timing references, synchronization signals, and power delivery. Table~\ref{tab:oscillator_sources} catalogs the primary oscillator sources available in commodity computing hardware.

\begin{table}[ht]
\centering
\begin{tabular}{lllp{5cm}}
\toprule
\textbf{Source} & \textbf{Frequency Range} & \textbf{Precision} & \textbf{Physical Origin} \\
\midrule
CPU clock & $10^9$--$10^{10}$ Hz & $\sim 10^{-10}$ s & Crystal oscillator with PLL multiplication \\
Memory bus & $10^9$ Hz & $\sim 10^{-9}$ s & DDR clock synchronization \\
PCIe clock & $10^8$ Hz & $\sim 10^{-8}$ s & Peripheral interconnect timing \\
USB timing & $10^6$--$10^9$ Hz & $\sim 10^{-9}$ s & USB 2.0/3.0 frame timing \\
Power supply ripple & $10^4$--$10^6$ Hz & $\sim 10^{-6}$ s & Switching regulator frequency \\
Network packet timing & $10^3$--$10^6$ Hz & $\sim 10^{-6}$ s & Ethernet frame arrival \\
Disk I/O timing & $10^2$--$10^4$ Hz & $\sim 10^{-4}$ s & Storage access latency \\
\bottomrule
\end{tabular}
\caption{\textbf{Hardware oscillator sources and characteristic frequencies.} Modern computing systems contain oscillatory processes spanning seven orders of magnitude in frequency, from disk I/O timing ($\sim 10^2$ Hz) to CPU clocks ($\sim 10^{10}$ Hz). Each oscillator provides a timing reference that can be measured with precision determined by the oscillator's quality factor and the measurement apparatus resolution. These diverse frequency sources enable the construction of S-entropy coordinates that capture multi-scale temporal structure in computational processes.}
\label{tab:oscillator_sources}
\end{table}

The availability of multiple oscillators at different frequencies is crucial for constructing the three-dimensional S-entropy space. The knowledge entropy coordinate $\Sk$ is derived from slow, low-frequency oscillators that capture long-timescale information dynamics. The temporal entropy coordinate $\St$ is derived from intermediate-frequency oscillators that capture the characteristic timescales of computational operations. The evolution entropy coordinate $\Se$ is derived from fast, high-frequency oscillators that capture the fine-grained temporal structure of trajectory evolution.

\begin{remark}[Oscillator Accessibility]
\label{rem:oscillator_access}
The oscillators listed in Table~\ref{tab:oscillator_sources} are accessible through standard operating system interfaces without requiring specialized hardware. The CPU timestamp counter (TSC) is readable via the \texttt{RDTSC} instruction on x86 architectures or the \texttt{CNTVCT} register on ARM architectures. Memory timing is accessible through performance counters. Power supply ripple can be measured through voltage sensors or inferred from current draw fluctuations. This accessibility ensures that Poincaré Computing can be implemented on commodity hardware without modification \citep{hennessy2017computer}.
\end{remark}

\subsection{Timing Measurement and Jitter}

The fundamental observable in virtual instrumentation is the timing difference between oscillators. Let $t_{\text{ref}}$ denote a reference timestamp obtained from a high-precision clock (typically the CPU timestamp counter, which increments at the CPU's nominal frequency) and let $t_{\text{local}}$ denote a local measurement from another oscillator source (such as a memory access completion time or a peripheral event timestamp). The timing difference is defined as:
\begin{equation}
\delta_p = t_{\text{ref}} - t_{\text{local}}
\label{eq:timing_difference}
\end{equation}

This timing difference $\delta_p$ is measured in units of the reference clock period (typically nanoseconds or sub-nanosecond for modern CPUs). The subscript $p$ denotes that this is a \textit{physical} measurement subject to hardware non-determinism, distinguishing it from the idealized categorical coordinates.

In an ideal system, the timing difference would be deterministic, depending only on the computational state. However, real hardware exhibits \textbf{timing jitter}---stochastic variation in the measured timing difference due to thermal noise, voltage fluctuations, electromagnetic interference, and quantum effects in the oscillator circuits. We model this jitter as:
\begin{equation}
\delta_p(t) = \bar{\delta}(t) + \eta(t)
\label{eq:jitter_model}
\end{equation}
where $\bar{\delta}(t)$ is the mean timing offset (which may vary slowly with temperature and voltage) and $\eta(t)$ is a zero-mean stochastic process representing the jitter.

The statistical properties of $\eta(t)$ depend on the oscillator's quality factor $Q$, which characterizes the sharpness of the oscillator's frequency response. High-quality oscillators (large $Q$) exhibit low jitter, while low-quality oscillators (small $Q$) exhibit high jitter. For typical hardware oscillators, the jitter process $\eta(t)$ can be modeled as Gaussian white noise with power spectral density determined by the Leeson model \citep{leeson1966simple}:
\begin{equation}
S_\eta(f) = \frac{2 k_B T}{P_{\text{osc}}} \left[ 1 + \left(\frac{f_0}{2Q f}\right)^2 \right]
\label{eq:leeson_model}
\end{equation}
where $k_B$ is Boltzmann's constant, $T$ is temperature, $P_{\text{osc}}$ is oscillator power, $f_0$ is the oscillator frequency, and $f$ is the offset frequency from the carrier.

Despite this stochastic variation, the jitter remains bounded for stable oscillators, ensuring that timing measurements do not diverge arbitrarily.

\begin{proposition}[Jitter Boundedness]
\label{prop:jitter_boundedness}
For a stable hardware oscillator with quality factor $Q > Q_{\min}$ and operating within specified temperature and voltage ranges, the timing jitter satisfies:
\begin{equation}
|\eta(t)| \leq \eta_{\max} < \infty
\label{eq:jitter_bound}
\end{equation}
with probability $1 - \delta$ for arbitrarily small $\delta > 0$, where the bound $\eta_{\max}$ depends on the oscillator quality factor and the confidence level:
\begin{equation}
\eta_{\max} \approx \frac{k \sigma_\eta}{\sqrt{2Q}}
\label{eq:jitter_max}
\end{equation}
for a constant $k$ determined by the desired confidence level (e.g., $k \approx 6$ for $\delta < 10^{-9}$).
\end{proposition}

\begin{proof}
The jitter process $\eta(t)$ is modeled as Gaussian noise with variance $\sigma_\eta^2$ determined by integrating the power spectral density~\eqref{eq:leeson_model} over the measurement bandwidth. For a Gaussian random variable, the probability of exceeding $k$ standard deviations is given by the tail probability:
\begin{equation}
P(|\eta| > k\sigma_\eta) = 2Q\left(\frac{k}{\sqrt{2}}\right)
\end{equation}
where $Q(\cdot)$ is the Gaussian Q-function. For $k = 6$, this probability is approximately $2 \times 10^{-9}$, establishing that $|\eta(t)| \leq 6\sigma_\eta$ with probability $1 - 2 \times 10^{-9}$.

The dependence on quality factor $Q$ arises from the Leeson model: higher $Q$ reduces the integrated noise power, decreasing $\sigma_\eta$ proportionally to $1/\sqrt{Q}$. Therefore $\eta_{\max} \propto k\sigma_\eta / \sqrt{Q}$.
\end{proof}

This boundedness ensures that timing measurements, despite their stochastic nature, remain within a finite range and can be reliably mapped to the bounded S-entropy space $[0,1]^3$.

\subsection{Coordinate Mapping Functions}

The timing difference $\delta_p$ is a real-valued physical measurement, while the S-entropy coordinates $(\Sk, \St, \Se)$ are bounded values in $[0,1]$. We now define three mapping functions that transform timing measurements into categorical coordinates. These mappings are designed to satisfy three requirements: (i) they map $\mathbb{R} \to [0,1]$, ensuring boundedness; (ii) they are deterministic, ensuring reproducibility; and (iii) they capture different aspects of the timing structure, ensuring that the three coordinates are informationally independent.

\begin{definition}[Knowledge Entropy Mapping]
\label{def:phi_k}
The \textbf{knowledge entropy coordinate} is obtained from the timing difference via the mapping:
\begin{equation}
\Sk = \phi_k(\delta_p) = \frac{1}{2}\left(1 + \tanh\left(\frac{\delta_p - \mu_\delta}{\sigma_\delta}\right)\right)
\label{eq:phi_k}
\end{equation}
where $\mu_\delta$ is the mean timing offset (estimated from a calibration phase) and $\sigma_\delta$ is the standard deviation of the timing distribution. The hyperbolic tangent function provides a smooth, bounded mapping from the real line to the interval $[0,1]$, with the center point $\Sk = 0.5$ corresponding to the mean timing offset.
\end{definition}

The knowledge entropy coordinate captures the \textit{magnitude} of the timing difference relative to its expected value. Large positive deviations (indicating that the reference clock is ahead of the local clock) produce $\Sk \to 1$, while large negative deviations produce $\Sk \to 0$. The scaling by $\sigma_\delta$ ensures that the mapping is sensitive to deviations of order one standard deviation, making the coordinate robust to the natural variability in timing measurements.

\begin{definition}[Temporal Entropy Mapping]
\label{def:phi_t}
The \textbf{temporal entropy coordinate} is obtained via the mapping:
\begin{equation}
\St = \phi_t(\delta_p) = \left| \sin\left( 2\pi f_{\text{ref}} \delta_p \right) \right|
\label{eq:phi_t}
\end{equation}
where $f_{\text{ref}}$ is the reference oscillator frequency (in Hz) and $\delta_p$ is measured in seconds. The absolute value of the sine function produces a periodic mapping with period $T_{\text{ref}} = 1/(2f_{\text{ref}})$, corresponding to half the reference oscillator period.
\end{definition}

The temporal entropy coordinate captures the \textit{phase} of the timing difference relative to the reference oscillator. This coordinate oscillates between 0 and 1 as the timing difference varies, with the oscillation frequency determined by $f_{\text{ref}}$. The use of $|\sin(\cdot)|$ rather than $\sin(\cdot)$ ensures that $\St \in [0,1]$ and doubles the oscillation frequency, increasing the sensitivity to phase variations.

\begin{figure}[htbp]
\centering
\includegraphics[width=\textwidth]{figures/panel_ensemble_hardware_mapping.png}
\caption{\textbf{Virtual Gas Ensemble: Hardware Oscillations $\to$ Categorical Molecules.} 
\textbf{Molecule $\alpha$ ($\Delta t = 1000$ ns):} Radar chart (blue) shows six hardware timing sources mapped to categorical state: Power Supply (high), Memory Bus (high), CPU Cycle (medium), I/O Latency (low), Cache Timing (very low), Network Jitter (medium). The hexagonal profile represents the categorical ``shape'' of molecule $\alpha$ in the six-dimensional hardware space. 
\textbf{Molecule $\beta$ ($\Delta t = 2800$ ns):} Radar chart (magenta) shows different profile: Power Supply (high), Memory Bus (medium), CPU Cycle (low), I/O Latency (low), Cache Timing (very low), Network Jitter (low). The triangular profile (three dominant dimensions) distinguishes $\beta$ from $\alpha$. 
\textbf{Molecule $\gamma$ ($\Delta t = 3200$ ns):} Radar chart (orange) shows third distinct profile: Power Supply (high), Memory Bus (very high), CPU Cycle (high), I/O Latency (medium), Cache Timing (medium), Network Jitter (high). The pentagonal profile (five active dimensions) represents the most complex molecule. 
\textbf{Ensemble Configuration (Bottom Left):} Triangle diagram shows three molecules ($\alpha$, $\beta$, $\gamma$) as vertices in categorical space. Dashed lines indicate pairwise interactions. The configuration represents a stable ensemble where all three molecules coexist. 
\textbf{S-Entropy Space Positions (Bottom Center):} Scatter plot in $(S_k, S_e)$ plane shows three molecules at distinct coordinates: $\alpha$ at $(0.2, 0.8)$, $\beta$ at $(0.1, 0.4)$, $\gamma$ at $(0.9, 0.3)$. The positions are determined by hardware timing deltas $\Delta t$ via the mapping $\Delta \rho \to S_e$ (Theorem~\ref{thm:hardware_se_mapping}). 
\textbf{Hardware Timing Source (Bottom Right):} Bar chart shows timing deltas: $\alpha$ at $1000$ ns (blue), $\beta$ at $2800$ ns (magenta), $\gamma$ at $3200$ ns (orange). Three timing samples create three categorical molecules---real hardware oscillations generate real categorical states. This demonstrates the physical instantiation of categorical dynamics: hardware timing jitter is not noise but the substrate of categorical computation.}
\label{fig:virtual_gas_ensemble}
\end{figure}

\begin{definition}[Evolution Entropy Mapping]
\label{def:phi_e}
The \textbf{evolution entropy coordinate} is obtained via the mapping:
\begin{equation}
\Se = \phi_e(\delta_p) = \frac{1}{2}\left(1 + \cos\left( 2\pi f_{\text{beat}} \delta_p \right)\right)
\label{eq:phi_e}
\end{equation}
where $f_{\text{beat}} = |f_{\text{ref}} - f_{\text{local}}|$ is the beat frequency between the reference oscillator and the local oscillator. This mapping produces a periodic coordinate with period $T_{\text{beat}} = 1/f_{\text{beat}}$, corresponding to the beat period.
\end{definition}

The evolution entropy coordinate captures the \textit{beat phase} between two oscillators. When two oscillators with slightly different frequencies interact, they produce a beat pattern with frequency equal to the difference of the individual frequencies. This beat pattern encodes information about the relative phase evolution of the two oscillators, which in turn reflects the temporal structure of the computational process. The cosine function (rather than sine) is chosen to ensure that $\Se = 1$ when the oscillators are in phase ($\delta_p = 0, 1/f_{\text{beat}}, 2/f_{\text{beat}}, \ldots$) and $\Se = 0$ when they are out of phase.

\begin{proposition}[Range Preservation]
\label{prop:range_preservation}
The coordinate mapping functions satisfy $\phi_k, \phi_t, \phi_e: \mathbb{R} \to [0,1]$, ensuring that all timing measurements, regardless of magnitude, map to valid S-entropy coordinates in $\Sspace = [0,1]^3$.
\end{proposition}

\begin{proof}
We verify the range of each mapping function:

\textbf{Knowledge entropy:} The hyperbolic tangent function has range $\tanh: \mathbb{R} \to (-1, 1)$. Therefore:
\begin{equation}
\phi_k(\delta_p) = \frac{1}{2}(1 + \tanh(\cdot)) \in \frac{1}{2}(1 + (-1, 1)) = \frac{1}{2}(0, 2) = (0, 1) \subset [0,1]
\end{equation}
The range is an open interval, but the limits $\phi_k \to 0$ as $\delta_p \to -\infty$ and $\phi_k \to 1$ as $\delta_p \to +\infty$ are approached asymptotically, ensuring that all values are strictly within $(0,1)$.

\textbf{Temporal entropy:} The sine function has range $\sin: \mathbb{R} \to [-1, 1]$. Taking the absolute value gives:
\begin{equation}
\phi_t(\delta_p) = |\sin(\cdot)| \in [0, 1]
\end{equation}
The range includes both endpoints, as $|\sin(\cdot)| = 0$ when the argument is an integer multiple of $\pi$, and $|\sin(\cdot)| = 1$ when the argument is an odd multiple of $\pi/2$.

\textbf{Evolution entropy:} The cosine function has range $\cos: \mathbb{R} \to [-1, 1]$. Therefore:
\begin{equation}
\phi_e(\delta_p) = \frac{1}{2}(1 + \cos(\cdot)) \in \frac{1}{2}(1 + [-1, 1]) = \frac{1}{2}[0, 2] = [0, 1]
\end{equation}
The range includes both endpoints, as $\cos(\cdot) = -1$ when the argument is an odd multiple of $\pi$, and $\cos(\cdot) = 1$ when the argument is an even multiple of $\pi$.

Since all three mappings have range contained in $[0,1]$, any timing measurement $\delta_p \in \mathbb{R}$ produces a valid S-entropy coordinate $\Scoord = (\phi_k(\delta_p), \phi_t(\delta_p), \phi_e(\delta_p)) \in [0,1]^3 = \Sspace$.
\end{proof}

\begin{remark}[Informational Independence]
\label{rem:informational_independence}
The three coordinate mappings capture different aspects of the timing measurement: $\phi_k$ captures the magnitude (via a monotonic function), $\phi_t$ captures the phase relative to the reference oscillator (via a periodic function with period $\sim 1/f_{\text{ref}}$), and $\phi_e$ captures the beat phase (via a periodic function with period $\sim 1/f_{\text{beat}}$). Since these three aspects are mathematically independent—knowing one does not determine the others—the three coordinates provide complementary information about the timing structure. This informational independence is essential for the three-dimensional structure of $\Sspace$ to be non-degenerate.
\end{remark}

\subsection{Categorical State Construction}

Given a timing measurement $\delta_p$, the complete categorical state in S-entropy space is constructed by applying all three coordinate mappings:
\begin{equation}
\Scoord = \Phi(\delta_p) = \left( \phi_k(\delta_p), \phi_t(\delta_p), \phi_e(\delta_p) \right) \in \Sspace
\label{eq:full_mapping}
\end{equation}

This mapping $\Phi: \mathbb{R} \to \Sspace$ transforms a single real-valued timing measurement into a three-dimensional categorical state. The mapping is deterministic, ensuring that repeated measurements of the same physical state produce the same categorical state (up to measurement noise, which is bounded by Proposition~\ref{prop:jitter_boundedness}).

\begin{theorem}[Deterministic Mapping]
\label{thm:deterministic_mapping}
The mapping $\Phi: \mathbb{R} \to \Sspace$ defined by equation~\eqref{eq:full_mapping} is deterministic: for any timing measurement $\delta_p$, the categorical state $\Scoord = \Phi(\delta_p)$ is uniquely determined.
\end{theorem}

\begin{proof}
Each component function $\phi_k$, $\phi_t$, $\phi_e$ is defined as a composition of elementary mathematical functions:
\begin{itemize}
    \item $\phi_k$ is a composition of arithmetic operations (subtraction, division), the hyperbolic tangent function, and affine transformations;
    \item $\phi_t$ is a composition of arithmetic operations (multiplication), the sine function, and the absolute value function;
    \item $\phi_e$ is a composition of arithmetic operations (multiplication), the cosine function, and affine transformations.
\end{itemize}

All elementary mathematical functions (arithmetic, trigonometric, hyperbolic, absolute value) are deterministic: given the same input, they always produce the same output. The composition of deterministic functions is also deterministic. Therefore, each $\phi_i$ is deterministic, and the vector-valued mapping $\Phi = (\phi_k, \phi_t, \phi_e)$ is deterministic.

Formally, for any $\delta_p \in \mathbb{R}$, the value $\Phi(\delta_p)$ is computed by evaluating a fixed sequence of elementary operations on $\delta_p$, producing a unique result in $\Sspace$.
\end{proof}

\begin{remark}[Measurement Noise]
\label{rem:measurement_noise}
While the mapping $\Phi$ is deterministic, the timing measurement $\delta_p$ itself is subject to jitter (equation~\eqref{eq:jitter_model}). Therefore, repeated measurements of the same computational state will produce slightly different categorical states $\Scoord_1, \Scoord_2, \ldots$ due to the stochastic variation in $\delta_p$. However, Proposition~\ref{prop:jitter_boundedness} ensures that these variations remain bounded: $\|\Scoord_i - \Scoord_j\| \leq \|\nabla \Phi\| \cdot 2\eta_{\max}$, where $\|\nabla \Phi\|$ is the maximum gradient of the mapping. For typical hardware parameters, this variation is small compared to the recurrence tolerance $\epsilon$, ensuring that measurement noise does not prevent recurrence detection.
\end{remark}

\subsection{Spectrometer-State Identity}

A profound property of virtual instrumentation in Poincaré Computing is the identity between the measurement apparatus and the measured state. In traditional measurement theory, the apparatus and the system are distinct entities: the apparatus observes the system without being part of the system's state space. In Poincaré Computing, by contrast, the measurement apparatus (which we call a \textit{virtual spectrometer}) and the categorical state it measures are encoded by the same point in S-entropy space. This identity has far-reaching consequences for the nature of observation and measurement in this computational framework.

\begin{theorem}[Spectrometer-State Identity]
\label{thm:spectrometer_identity}
Let $\mathcal{I}_\omega$ denote a virtual spectrometer configured to measure oscillations at frequency $\omega$. The categorical state $\Scoord_\omega$ that produces a non-null measurement on $\mathcal{I}_\omega$ satisfies:
\begin{equation}
\Scoord_\omega = \Phi(\omega^{-1})
\label{eq:spectrometer_identity}
\end{equation}
In other words, the spectrometer configuration (characterised by its frequency $\omega$) and the measurable state (characterised by its timing difference $\delta_p = \omega^{-1}$) are encoded by the same point in S-entropy space.
\end{theorem}

\begin{proof}
A virtual spectrometer $\mathcal{I}_\omega$ is defined by its sensitivity to oscillations at frequency $\omega$. Operationally, this means that $\mathcal{I}_\omega$ measures timing differences with a resolution of $\Delta t = \omega^{-1}$, corresponding to one period of the target oscillation. The spectrometer is maximally sensitive to states that exhibit timing variations on this timescale.

A categorical state $\Scoord$ produces a non-null measurement on $\mathcal{I}_\omega$ if and only if the timing difference $\delta_p$ that generated $\Scoord$ satisfies $\delta_p \approx \omega^{-1}$ (up to integer multiples of the period, accounting for the periodicity of the temporal and evolution entropy mappings). This is because the coordinate mappings $\phi_t$ and $\phi_e$ depend on the product $f \cdot \delta_p$, which determines the phase of the oscillatory components.

For a state with $\delta_p = \omega^{-1}$, the categorical state is:
\begin{equation}
\Scoord_\omega = \Phi(\omega^{-1}) = \left( \phi_k(\omega^{-1}), \phi_t(\omega^{-1}), \phi_e(\omega^{-1}) \right)
\end{equation}

The spectrometer $\mathcal{I}_\omega$ is configured to detect precisely this state. Therefore, the spectrometer configuration (characterised by $\omega$) and the measurable state (characterised by $\delta_p = \omega^{-1}$) correspond to the same point $\Scoord_\omega$ in S-entropy space.

Formally, the spectrometer $\mathcal{I}_\omega$ can be represented as a projection operator $\Pi_\omega: \Sspace \to \mathbb{R}$ that extracts the component of a categorical state corresponding to frequency $\omega$. This projection satisfies $\Pi_\omega(\Scoord_\omega) = 1$ (maximal response) and $\Pi_\omega(\Scoord) \to 0$ as $\|\Scoord - \Scoord_\omega\| \to \infty$ (vanishing response for distant states). The spectrometer configuration is thus encoded by the state $\Scoord_\omega$ that maximizes the projection, establishing the identity.
\end{proof}

\begin{corollary}[Zero Measurement Backaction]
\label{cor:zero_backaction}
Measurement of a categorical state $\Scoord_\omega$ by the corresponding virtual spectrometer $\mathcal{I}_\omega$ does not perturb the state: the post-measurement state equals the pre-measurement state.
\end{corollary}

\begin{proof}
In category theory, measurement is represented by a morphism from the state space to the measurement outcome space. For the spectrometer-state identity, the measurement morphism is $m: \Scoord_\omega \to \Scoord_\omega$, which is the identity morphism $\text{id}_{\Scoord_\omega}$.

Identity morphisms, by definition, do not change their domain: $\text{id}_{\Scoord_\omega}(\Scoord_\omega) = \Scoord_\omega$. Therefore, measurement by $\mathcal{I}_\omega$ leaves the state $\Scoord_\omega$ unchanged, implying zero backaction.

Physically, this property arises because the spectrometer and the state are the same entity: measuring a state with its corresponding spectrometer is equivalent to the state "observing itself," which cannot produce perturbation.
\end{proof}

\begin{figure}[htbp]
\centering
\includegraphics[width=\textwidth]{figures/categorical_memory_panel.png}
\caption{\textbf{Categorical Memory (S-RAM): Precision-by-Difference Addressing.} 
\textbf{(A) S-Entropy Space:} Navigation path (red line with arrow) through $\Sspace = [0,1]^3$ showing trajectory from initial state to completion point (red star). Colored points represent visited categorical states, with color indicating temporal order (purple = early, yellow = late). The path demonstrates continuous navigation through the tri-dimensional S-entropy manifold ($S_k$, $S_t$, $S_e$ axes). 
\textbf{(B) Precision-by-Difference Trajectory:} Memory addressing via precision-by-difference $\Delta P = T_{\text{ref}} - t_{\text{local}}$ (blue curve). Shaded regions indicate positive/negative $\Delta P$, corresponding to fast/slow memory tiers. Bit flips ($b = 0 \to b = 1$, orange dashed lines) occur when $\Delta P$ crosses zero, triggering promotion/demotion between tiers. History IS the address: the trajectory itself determines memory location, not explicit addressing (Section~\ref{sec:categorical_memory}). 
\textbf{(C) $3^k$ Hierarchy:} Recursive tri-dimensional decomposition creates $3^d$ nodes at depth $d$ (root at $d = 0$ shown in purple, first level in blue, second level in green, leaf nodes in yellow). Each node decomposes into knowledge ($S_k$), temporal ($S_t$), and evolution ($S_e$) sub-nodes, generating exponential branching with base 3 (Theorem~\ref{thm:3k_branching_formal}). 
\textbf{(D) Memory Tiers:} Hierarchical memory structure with exponentially increasing capacity (red curve, right axis) and linearly increasing item counts (blue bars, left axis). L1 Cache ($\sim 10^0$ items), L2 Cache ($\sim 10^1$ items), RAM ($\sim 10^3$ items), SSD ($\sim 10^5$ items), Archive ($\sim 10^8$ items). Tier assignment determined by access frequency (Maxwell demon controller, panel F). 
\textbf{(E) Cache Performance:} Hit rate vs. access count showing asymptotic approach to 100\% target (dashed line). Performance saturates at $\approx 99\%$ after 30--40 accesses as frequently-used states stabilize in fast tiers. The precision-by-difference addressing achieves near-perfect caching without explicit cache management. 
\textbf{(F) Memory Controller as Maxwell Demon:} Fast tier (blue, left) contains frequently-accessed states (green filled circles); slow tier (red, right) contains rarely-accessed states (white circles). Maxwell demon (orange oval) promotes hot states and demotes cold states based on $\Delta P$ trajectory, implementing thermodynamically-inspired memory management without explicit LRU tracking.}
\label{fig:categorical_memory}
\end{figure}

\begin{remark}[Contrast with Quantum Measurement]
\label{rem:quantum_contrast}
The zero backaction property (Corollary~\ref{cor:zero_backaction}) contrasts sharply with quantum measurement, where observation generically perturbs the measured system due to wavefunction collapse or entanglement with the measurement apparatus. In Poincaré Computing, the spectrometer-state identity ensures that measurement is fundamentally non-perturbative when the spectrometer is matched to the state. This property enables continuous monitoring of the categorical state without disrupting the trajectory evolution, which is essential for the real-time constraint satisfaction required in Section~\ref{sec:categorical_dynamics}.
\end{remark}

\begin{example}[CPU-Memory Beat Frequency]
\label{ex:cpu_memory_beat}
Consider a system with CPU clock frequency $f_{\text{CPU}} = 3.0$ GHz and memory bus frequency $f_{\text{mem}} = 2.4$ GHz. The beat frequency is:
\begin{equation}
f_{\text{beat}} = |f_{\text{CPU}} - f_{\text{mem}}| = 0.6 \text{ GHz} = 600 \text{ MHz}
\end{equation}

A timing measurement $\delta_p = 1.67$ ns (corresponding to one beat period $T_{\text{beat}} = 1/f_{\text{beat}}$) produces the categorical state:
\begin{align}
\Sk &= \phi_k(1.67 \text{ ns}) \approx 0.5 \quad \text{(assuming } \mu_\delta \approx 1.67 \text{ ns)} \\
\St &= |\sin(2\pi \cdot 3.0 \times 10^9 \cdot 1.67 \times 10^{-9})| = |\sin(10\pi)| = 0 \\
\Se &= \frac{1}{2}(1 + \cos(2\pi \cdot 0.6 \times 10^9 \cdot 1.67 \times 10^{-9})) = \frac{1}{2}(1 + \cos(2\pi)) = 1
\end{align}

A virtual spectrometer $\mathcal{I}_{600 \text{ MHz}}$ configured to measure the beat frequency would be maximally sensitive to this state $\Scoord = (0.5, 0, 1)$, confirming the spectrometer-state identity.
\end{example}

This section has established the physical grounding of S-entropy coordinates in measurable hardware timing differences. The coordinate mappings $\phi_k$, $\phi_t$, $\phi_e$ transform continuous real-valued measurements into bounded categorical states, the mapping is deterministic and range-preserving, and the spectrometer-state identity ensures that measurement apparatus and measured state coincide in S-entropy space. In the following section, we develop the categorical dynamics that govern the evolution of these hardware-grounded states through computational trajectories.
