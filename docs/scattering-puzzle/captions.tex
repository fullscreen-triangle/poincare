% Captions for Refraction Puzzle Imaging (RPI) Figures
% Generated for turbine/publication/scattering-puzzle/experiments/figures/

%% ============================================================================
%% FIGURE 1: Discrete Path Enumeration Validation
%% ============================================================================
\begin{figure*}[!htbp]
\centering
\includegraphics[width=\textwidth]{experiments/figures/panel_discrete_paths.pdf}
\caption{\textbf{Discrete Path Enumeration Validation.}
\textbf{Panel A (top-left):} Three-dimensional visualization of discrete path probability distribution across spatial coordinates $X$ and $Y$, showing characteristic peaked structure with path density values reaching 0.20 at central maxima. The discrete combinatorial framework generates well-defined probability landscapes that encode all valid light trajectories through the scattering medium.
\textbf{Panel B (top-right):} Convergence analysis demonstrating mean squared error (MSE) between discrete RPI and continuous wave optics as a function of discrete direction count. The MSE stabilizes at approximately $8 \times 10^{-3}$ beyond 50 directions, confirming rapid convergence to the wave-optical limit with modest discretization.
\textbf{Panel C (bottom-left):} Path-wave correlation matrix showing agreement between discrete path enumeration and wave propagation across varying propagation steps (5--25) and discrete directions (12--251). Correlation values increase systematically from 0.66 to 0.87, demonstrating improved fidelity with finer discretization.
\textbf{Panel D (bottom-right):} Entropy evolution during propagation for different discretization levels (6, 12, 25 directions). All curves remain well below the maximum entropy bound (dashed line), indicating that discrete path enumeration preserves the constraint structure essential for unique reconstruction.}
\label{fig:discrete_paths}
\end{figure*}

%% ============================================================================
%% FIGURE 2: Transfer Matrix Analysis
%% ============================================================================
\begin{figure*}[!htbp]
\centering
\includegraphics[width=\textwidth]{experiments/figures/panel_transfer_matrix.pdf}
\caption{\textbf{Transfer Matrix Analysis for RPI Reconstruction.}
\textbf{Panel A (top-left):} Three-dimensional surface plot of effective matrix rank as a function of scattering strength and aberration level. The rank increases monotonically with scattering (0.2--1.0) while remaining relatively stable under aberration, reaching maximum values of approximately 60 at high scattering conditions. This demonstrates that scattering enhances rather than degrades reconstruction capability.
\textbf{Panel B (top-right):} Singular value spectrum on logarithmic scale for three scattering regimes: no scattering (blue), low scattering (orange), and high scattering (green). The rank threshold (red dashed line) indicates the cutoff for significant singular values. High scattering maintains elevated singular values across a broader index range, directly translating to improved reconstruction fidelity.
\textbf{Panel C (bottom-left):} Reconstruction error (MSE) heatmap as a function of scattering and aberration strength. Green regions (low error, MSE $\sim$0.2) dominate at high scattering, while red regions (high error, MSE $\sim$0.8) appear only at low scattering with strong aberration, quantifying the protective effect of scattering against optical degradation.
\textbf{Panel D (bottom-right):} Matrix condition number versus scattering strength for ideal optics (black) and aberrated systems (gray shaded region). Both decrease exponentially with scattering, with condition numbers dropping from $10^7$ to below $10^5$, indicating dramatically improved numerical stability for reconstruction.}
\label{fig:transfer_matrix}
\end{figure*}

%% ============================================================================
%% FIGURE 3: Scattering Enhancement of Reconstruction
%% ============================================================================
\begin{figure*}[!htbp]
\centering
\includegraphics[width=\textwidth]{experiments/figures/panel_scattering_enhancement.pdf}
\caption{\textbf{Scattering Enhancement of Reconstruction Quality.}
\textbf{Panel A (top-left):} Three-dimensional surface showing reconstruction quality (PSNR in dB) as a function of image complexity and number of scatterers. Quality increases from approximately 18~dB to 24~dB as scatterer count rises from 0 to 500, with the enhancement effect more pronounced for complex images. This counter-intuitive result demonstrates the fundamental RPI principle that scattering improves imaging.
\textbf{Panel B (top-right):} Fluorescence microscopy demonstration of scatter-then-reconstruct protocol. Original cellular image (left) undergoes controlled scattering (center), followed by RPI reconstruction (right). The reconstructed image recovers fine cellular detail despite passage through the scattering medium.
\textbf{Panel C (bottom-left):} Quantitative reconstruction metrics versus scatterer count. PSNR (red circles, left axis) increases from 13~dB to 20~dB, while SSIM (black squares, right axis) improves from 0.68 to 0.92. Both metrics exhibit logarithmic saturation behavior, indicating optimal scatterer densities for practical implementation.
\textbf{Panel D (bottom-right):} Effective rank enhancement as a function of scatterer count. The transfer matrix rank increases from 32 (no scattering) to 58 (500 scatterers), surpassing the classical diffraction limit (orange dashed line at rank 60). This rank enhancement directly enables super-resolution reconstruction capabilities.}
\label{fig:scattering_enhancement}
\end{figure*}

%% ============================================================================
%% FIGURE 4: Phase Discretization Analysis
%% ============================================================================
\begin{figure*}[!htbp]
\centering
\includegraphics[width=\textwidth]{experiments/figures/panel_phase_discretization.pdf}
\caption{\textbf{Phase Discretization Analysis for RPI.}
\textbf{Panel A (top-left):} Three-dimensional visualization of discretized phase distribution across spatial coordinates, showing characteristic oscillatory structure with phase values ranging from $-0.4$ to $+0.3$ radians. The discrete phase representation captures the essential wave-optical information required for reconstruction while enabling combinatorial constraint satisfaction.
\textbf{Panel B (top-right):} Error scaling analysis on log-log axes showing both phase error (orange) and reconstruction error (black) versus number of phase discretization levels. Both errors decrease as power laws, with reconstruction error dropping from $2 \times 10^{-1}$ to $3 \times 10^{-2}$ as phase levels increase from 8 to 128.
\textbf{Panel C (bottom-left):} Phase wrapping resolution comparison showing true phase (black solid), wrapped phase (orange dashed), and discrete 16-level approximation (green dotted). The discrete representation faithfully tracks the true phase while avoiding the $2\pi$ discontinuities that plague continuous phase unwrapping algorithms.
\textbf{Panel D (bottom-right):} Phase wrap reduction achieved through discretization. Bar chart comparing continuous (coral) versus discrete (teal) phase representations across discretization levels (4--128). Discrete representations consistently exhibit 30--96\% fewer phase wraps, with the advantage increasing at finer discretization.}
\label{fig:phase_discretization}
\end{figure*}

%% ============================================================================
%% FIGURE 5: Aberration Invariance Under RPI
%% ============================================================================
\begin{figure*}[!htbp]
\centering
\includegraphics[width=\textwidth]{experiments/figures/panel_aberration_invariance.pdf}
\caption{\textbf{Aberration Invariance Under RPI Reconstruction.}
\textbf{Panel A (top-left):} Three-dimensional visualization of an aberrated point spread function (PSF) showing characteristic asymmetric broadening. The PSF intensity peaks at 1.0 at center with extended tails along both $X$ and $Y$ directions, representing combined defocus, astigmatism, and coma aberrations typical of imperfect optical systems.
\textbf{Panel B (top-right):} Recovery quality comparison across aberration types. Orange bars show PSNR for aberrated images (10--18~dB), while green bars show RPI-reconstructed quality (28--35~dB). Improvement is consistent across perfect optics, defocus, astigmatism, coma, spherical aberration, and mixed aberrations, demonstrating universal aberration correction.
\textbf{Panel C (bottom-left):} Microscopy aberration correction demonstration. Original cellular image (left) degraded by aberrations (center) is recovered through RPI reconstruction (right). Fine subcellular structures obscured by aberration are clearly resolved in the reconstruction.
\textbf{Panel D (bottom-right):} PSNR improvement versus aberration strength. The improvement (teal curve with confidence band) increases from 0~dB at zero aberration to approximately 15~dB at maximum aberration strength (2.0), asymptotically approaching the theoretical maximum recovery limit (dashed line). This confirms that RPI provides greater benefit precisely when aberrations are most severe.}
\label{fig:aberration_invariance}
\end{figure*}

%% ============================================================================
%% FIGURE 6: Computational Complexity Analysis
%% ============================================================================
\begin{figure*}[!htbp]
\centering
\includegraphics[width=\textwidth]{experiments/figures/panel_computational_complexity.pdf}
\caption{\textbf{Computational Complexity Analysis of RPI.}
\textbf{Panel A (top-left):} Three-dimensional surface of computation time (log scale) as a function of image size and scattering strength. Computation time ranges from $10^{-3}$~s to $10^{-1}$~s, scaling smoothly with both parameters. The surface demonstrates predictable computational requirements for practical implementation planning.
\textbf{Panel B (top-right):} Computational scaling comparison on log-log axes. Measured RPI scaling (orange circles) follows $O(n^{1.5})$ complexity, substantially better than naive $O(n^3)$ approaches (black dashed) though above the FFT-based $O(n^2 \log n)$ lower bound (green dotted). This intermediate scaling enables processing of megapixel images in reasonable time.
\textbf{Panel C (bottom-left):} Computation time breakdown by algorithmic component. Matrix construction dominates at 35\% (orange), followed by SVD decomposition (dark blue), reconstruction (15\%, teal), and other operations (gray). This breakdown identifies optimization targets for future algorithmic improvements.
\textbf{Panel D (bottom-right):} Memory requirements versus image size for full matrix (orange) and sparse approximation (teal) approaches. Full matrix storage exceeds the 32~GB limit (gray dashed) beyond 96 pixels, while sparse approximation remains below 8~GB (black dashed) even at 128 pixels, enabling practical implementation on standard hardware.}
\label{fig:computational_complexity}
\end{figure*}

%% ============================================================================
%% FIGURE 7: Transplanckian Time Resolution Achievement
%% ============================================================================
\begin{figure*}[!htbp]
\centering
\includegraphics[width=\textwidth]{experiments/figures/panel_transplanckian_resolution.pdf}
\caption{\textbf{Transplanckian Time Resolution Achievement.}
\textbf{Panel A (top-left):} Three-dimensional visualization of the discrete light cone structure in the RPI framework. Path density peaks sharply at the origin with spatial index ($10^{-156}$~m scale) and temporal index ($10^{-156}$~s scale) axes, demonstrating the ultra-fine discretization that enables transplanckian resolution through combinatorial path enumeration.
\textbf{Panel B (top-right):} Temporal resolution hierarchy comparing RPI achievement to fundamental physical timescales. The horizontal bar chart spans from picosecond ($10^{-12}$~s) through femtosecond, attosecond, and Planck time ($10^{-44}$~s) to the RPI resolution regime ($10^{-156}$~s, deep red). RPI achieves temporal precision 112 orders of magnitude beyond the Planck scale.
\textbf{Panel C (bottom-left):} Path space scaling demonstrating exponential growth of discrete paths. RPI at $10^{-156}$~m resolution (orange) achieves $10^{90}$ paths for centimeter-scale samples, compared to $10^8$ paths for wave optics (teal dashed). The shaded region indicates the RPI enhancement factor over classical approaches.
\textbf{Panel D (bottom-right):} Resolution versus temporal precision demonstration on fluorescence microscopy data. Three panels show reconstruction at $10^{-15}$~s (femtosecond), $10^{-44}$~s (Planck scale), and $10^{-156}$~s (transplanckian) precision. Progressive enhancement of cellular detail demonstrates that finer temporal discretization translates directly to improved spatial reconstruction.}
\label{fig:transplanckian_resolution}
\end{figure*}

%% ============================================================================
%% FIGURE 8: Cellular Scattering Pattern Analysis
%% ============================================================================
\begin{figure*}[!htbp]
\centering
\includegraphics[width=\textwidth]{experiments/figures/panel_cellular_scattering.pdf}
\caption{\textbf{Cellular Scattering Pattern Analysis.}
\textbf{Panel A (top-left):} Three-dimensional cellular scattering map showing scattering coefficient $\mu_s$ (mm$^{-1}$) distribution across a cell. The surface exhibits characteristic peaks (reaching 3.5~mm$^{-1}$) corresponding to organelles and membrane structures, with valleys (0.5~mm$^{-1}$) in cytoplasmic regions. This heterogeneous scattering landscape provides the constraints exploited by RPI.
\textbf{Panel B (top-right):} Cellular compartment mapping from scattering analysis. Fluorescence microscopy image with nuclear regions (blue), cytoplasmic compartments (green), and membrane structures resolved through differential scattering signatures. Scale bar indicates 5~$\mu$m.
\textbf{Panel C (bottom-left):} Phase function $P(\theta)$ versus scattering angle for different cellular compartments on log scale. Cytosol ($g=0.9$, red) shows strong forward scattering, while nucleoplasm ($g=0.85$, orange), ER lumen ($g=0.7$, green), mitochondrial matrix ($g=0.6$, teal), and extracellular space ($g=0.55$, purple) exhibit progressively broader angular distributions. These distinct scattering signatures enable compartment-specific reconstruction.
\textbf{Panel D (bottom-right):} Optical properties by cellular component. Refractive index $n$ (dark bars, left axis: 1.35--1.45) and scattering coefficient $\mu_s$ (coral bars, right axis: 0.5--2.5~mm$^{-1}$) for nucleus, cytoplasm, membrane, mitochondria, ER, and ribosomes. The variation in optical properties across compartments provides the physical basis for RPI cellular imaging.}
\label{fig:cellular_scattering}
\end{figure*}

%% ============================================================================
%% FIGURE 9: Intrinsic Spectral Decomposition
%% ============================================================================
\begin{figure*}[!htbp]
\centering
\includegraphics[width=\textwidth]{experiments/figures/panel_spectral_decomposition.pdf}
\caption{\textbf{Intrinsic Spectral Decomposition: The Graduated Instrument.}
\textbf{Panel A (top-left):} Three-dimensional visualization of chromatic path separation showing path fraction versus wavelength (400--1000~nm) and normalized time. Different wavelengths (rendered in corresponding colors from violet to red) follow distinct trajectories due to dispersion, creating a natural ``graduated instrument'' that separates spectral components through differential path statistics.
\textbf{Panel B (top-right):} Wavelength-dependent optical properties. Refractive index $n$ (black curve, left axis) decreases from 1.485 to 1.455 across the visible-NIR range following Cauchy dispersion. Scattering coefficients for Rayleigh (red dashed) and Mie (dotted) regimes span four orders of magnitude, with $\lambda^{-4}$ and $\lambda^{-1}$ scaling respectively.
\textbf{Panel C (bottom-left):} Intrinsic spectral decomposition demonstration. Broadband fluorescence image (left) is decomposed into hyperspectral UV/Vis/NIR components (right) without external dispersive elements. The spectral separation emerges naturally from wavelength-dependent scattering path statistics.
\textbf{Panel D (bottom-right):} Spectral dependence of reconstruction quality across wavelength bands (UV through IR). Matrix rank (bars, left axis) decreases from 70 at UV to 55 at IR wavelengths due to reduced scattering at longer wavelengths. Reconstruction quality SSIM (line, right axis) tracks the rank, ranging from 0.95 to 0.72. Shaded region indicates the transition from Rayleigh-dominated to Mie-dominated scattering regimes.}
\label{fig:spectral_decomposition}
\end{figure*}

%% ============================================================================
%% FIGURE 10: Partition Extraction - Tracing vs. Subtraction
%% ============================================================================
\begin{figure*}[!htbp]
\centering
\includegraphics[width=\textwidth]{experiments/figures/panel_partition_extraction.pdf}
\caption{\textbf{Partition Extraction: Tracing vs. Subtraction Paradigm.}
\textbf{Panel A (top-left):} Three-dimensional state space visualization showing partition $P$ (red ellipsoid) embedded within total field $L$ (blue transparent volume). The boundary $\partial P$ (red wireframe surface) separates the partition from its complement. The key insight is that $\partial P$ contains sufficient information to reconstruct $P$ without computing $L \setminus P$ explicitly.
\textbf{Panel B (top-right):} Information content comparison between subtraction (coral/orange) and tracing (teal/green) paradigms across processing stages. Subtraction loses information progressively (total information drops from 1.0 to 0.2), while tracing preserves total information throughout (constant at 1.0). The partition-extracted information (dark teal) is recovered completely at output, demonstrating zero-loss extraction.
\textbf{Panel C (bottom-left):} Transfer matrix $A$ visualization with partition constraints. The heatmap shows path weights (0.2--1.0) connecting source indices to detector indices. The block structure $A_P$ (bounded region) encodes paths that remain within the partition, enabling direct extraction without background computation.
\textbf{Panel D (bottom-right):} Holographic extraction protocol demonstration. Source partition $P$ (leftmost) generates boundary encoding $\partial P$ (second panel), which produces the holographic record $\partial_r$ (third panel). Final reconstruction $P'$ (rightmost) recovers the original partition from boundary information alone, validating the principle that boundary tracing preserves complete partition information.}
\label{fig:partition_extraction}
\end{figure*}
