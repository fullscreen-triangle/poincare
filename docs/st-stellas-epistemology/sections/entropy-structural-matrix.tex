\section{The S-Entropy Structural Matrix}

\subsection{Three Fundamental Coordinates}

S-entropy space is spanned by three coordinates that capture the essential dimensions of any knowledge-seeking process. This formulation draws on information-theoretic foundations \cite{shannon1948, cover2006} and the connection between entropy and probability \cite{boltzmann1877, jaynes1957}:

\begin{definition}[S-Entropy Coordinates]
The S-entropy coordinates $(S_k, S_t, S_e)$ are defined as:
\begin{enumerate}
    \item $S_k$ (Knowledge): The information deficit---distance from complete knowledge of the target
    \item $S_t$ (Time): The temporal separation---duration until the solution is reached
    \item $S_e$ (Entropy): The thermodynamic accessibility---entropy distance from optimal configuration
\end{enumerate}
\end{definition}

These coordinates are not independent; they are related through the constraint:
\begin{equation}
S_{\text{total}} = \sqrt{S_k^2 + S_t^2 + S_e^2}
\end{equation}

where $S_{\text{total}}$ represents the total ``distance'' from the solution in S-entropy space.

\subsection{The 9-Fold Equivalence Structure}

By the Triple Equivalence Theorem, each S-coordinate can be expressed in three equivalent forms. This generates a $3 \times 3$ matrix:

\begin{table}[h]
\centering
\begin{tabular}{|l|c|c|c|}
\hline
\textbf{Coordinate} & \textbf{Oscillatory} & \textbf{Categorical} & \textbf{Partition} \\
\hline
$S_t$ (Time) & Period $T = 2\pi/\omega$ & Category count $M$ & Lag sum $\sum_a \tau_a$ \\
\hline
$S_k$ (Knowledge) & $\ln(A/A_0)$ & $\ln n$ & $\ln(1/s)$ \\
\hline
$S_e$ (Entropy) & $k_B \sum \ln(A_i/A_0)$ & $k_B M \ln n$ & $k_B \sum \ln(1/s_a)$ \\
\hline
\end{tabular}
\caption{The $3 \times 3$ S-Entropy Structural Matrix}
\end{table}

Each row represents the same physical quantity expressed three ways. Each column represents a consistent perspective applied to all three coordinates.

\subsection{Mathematical Formalization}

\begin{definition}[$3 \times 3$ Structural Matrix]
The S-entropy structural matrix $\mathbf{M}$ is defined as:
\begin{equation}
\mathbf{M} = \begin{pmatrix}
M_{t,\text{osc}} & M_{t,\text{cat}} & M_{t,\text{part}} \\
M_{k,\text{osc}} & M_{k,\text{cat}} & M_{k,\text{part}} \\
M_{e,\text{osc}} & M_{e,\text{cat}} & M_{e,\text{part}}
\end{pmatrix}
= \begin{pmatrix}
T & M & \sum\tau_a \\
\ln(A/A_0) & \ln n & \ln(1/s) \\
k_B \sum \ln A_i & k_B M \ln n & k_B \sum \ln(1/s_a)
\end{pmatrix}
\end{equation}
\end{definition}

\begin{theorem}[Row Equivalence]
Each row of $\mathbf{M}$ consists of three equivalent expressions:
\begin{align}
M_{t,\text{osc}} &= M_{t,\text{cat}} \cdot \langle\tau_p\rangle = M_{t,\text{part}} \\
M_{k,\text{osc}} &= M_{k,\text{cat}} = M_{k,\text{part}} \\
M_{e,\text{osc}} &= M_{e,\text{cat}} = M_{e,\text{part}}
\end{align}
\end{theorem}

\subsection{Dimensional Expressions}

Each S-coordinate admits a detailed triple expression:

\subsubsection{Time Coordinate $S_t$}

\begin{align}
S_t^{\text{(osc)}} &= T = \frac{2\pi}{\omega_{\text{fundamental}}} \quad \text{(period of fundamental oscillation)} \\
S_t^{\text{(cat)}} &= M = \text{number of categorical transitions to traverse} \\
S_t^{\text{(part)}} &= \sum_{a=1}^{k} \tau_a = \text{total partition lag across all apertures}
\end{align}

The equivalence relation:
\begin{equation}
T = M \cdot \langle\tau_p\rangle = \sum_a \tau_a
\end{equation}

\subsubsection{Knowledge Coordinate $S_k$}

\begin{align}
S_k^{\text{(osc)}} &= \ln\left(\frac{A_{\text{target}}}{A_{\text{current}}}\right) \quad \text{(amplitude distance in log space)} \\
S_k^{\text{(cat)}} &= \ln\left(\frac{n_{\text{target}}}{n_{\text{current}}}\right) \quad \text{(categorical depth ratio)} \\
S_k^{\text{(part)}} &= \ln\left(\frac{s_{\text{current}}}{s_{\text{target}}}\right) \quad \text{(selectivity ratio)}
\end{align}

The equivalence relation:
\begin{equation}
\ln\left(\frac{A}{A_0}\right) = \ln(n) = \ln\left(\frac{1}{s}\right)
\end{equation}

\subsubsection{Entropy Coordinate $S_e$}

\begin{align}
S_e^{\text{(osc)}} &= k_B \sum_i \ln\left(\frac{A_i}{A_0}\right) \quad \text{(phase space entropy)} \\
S_e^{\text{(cat)}} &= k_B M \ln n \quad \text{(categorical configuration entropy)} \\
S_e^{\text{(part)}} &= k_B \sum_a \ln\left(\frac{1}{s_a}\right) \quad \text{(aperture entropy)}
\end{align}

The equivalence relation:
\begin{equation}
S = k_B \sum_i \ln\left(\frac{A_i}{A_0}\right) = k_B M \ln n = k_B \sum_a \ln\left(\frac{1}{s_a}\right)
\end{equation}

\subsection{Operational Interpretation}

The $3 \times 3$ matrix provides nine equivalent handles for navigating S-entropy space:

\begin{itemize}
    \item To reduce $S_t$: Increase oscillation frequency, reduce category count, or minimize partition lags
    \item To reduce $S_k$: Match amplitude to target, increase categorical depth, or improve selectivity
    \item To reduce $S_e$: Constrain phase space, reduce categorical multiplicity, or increase aperture selectivity
\end{itemize}

Any of the nine expressions can serve as the operational definition; the triple equivalence guarantees consistent results.

\subsection{The S-Distance Metric}

Navigation through S-entropy space is measured by the S-distance:

\begin{definition}[S-Distance]
The S-distance between current state and target state is:
\begin{equation}
d_S = \sqrt{(S_k^{\text{target}} - S_k^{\text{current}})^2 + (S_t^{\text{target}} - S_t^{\text{current}})^2 + (S_e^{\text{target}} - S_e^{\text{current}})^2}
\end{equation}
\end{definition}

By the 9-fold equivalence, this distance can be computed using any consistent column of the matrix. The choice is one of convenience, not correctness.

\begin{theorem}[Path Independence]
The S-distance $d_S$ is invariant under the choice of representation (oscillatory, categorical, or partition). All three yield identical numerical values.
\end{theorem}

This path independence is the mathematical basis for the universality of S-navigation: regardless of which representation a sentient system uses, it is traversing the same S-entropy space.

