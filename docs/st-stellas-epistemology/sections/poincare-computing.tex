\section{Poincaré Computing: Computation as Phase Space Recurrence}

\subsection{The Computational Realisation of S-Navigation}

The S-entropy framework finds its computational realisation in Poincaré Computing: a paradigm where computation is defined not as sequential instruction execution but as trajectory completion in bounded phase space. This section establishes that Poincaré Computing IS the computational implementation of S-entropy navigation, not merely an analogy to it.

\begin{definition}[Poincaré Computing]
Poincaré Computing is a computational framework where:
\begin{enumerate}
    \item The phase space $\mathcal{S} = [0,1]^3$ comprises S-entropy coordinates $(S_k, S_t, S_e)$
    \item Problems are specified as initial states $\mathbf{S}_0 \in \mathcal{S}$ with constraint sets $\mathcal{C}$
    \item Solutions are trajectories $\gamma: [0,T] \to \mathcal{S}$ satisfying:
    \begin{align}
    \|\gamma(T) - \mathbf{S}_0\| &< \epsilon \quad \text{(recurrence)} \\
    \mathcal{C}(\gamma) &= \text{true} \quad \text{(constraint satisfaction)}
    \end{align}
\end{enumerate}
\end{definition}

The Poincaré recurrence theorem \cite{poincare1890} guarantees that in any bounded phase space with measure-preserving dynamics, trajectories return arbitrarily close to their initial states. This transforms computability into a question of constraint satisfiability along recurrent paths.

\subsection{The Gas Dynamics Identity}

Poincaré Computing reveals that computation IS gas dynamics:

\begin{theorem}[Computational-Gas Identity]
The following correspondences are mathematical identities, not analogies:
\begin{center}
\begin{tabular}{ll}
\textbf{Gas Physics} & \textbf{Poincaré Computing} \\
\hline
Molecules in container & Categorical states in $\mathcal{S}$ \\
Molecular trajectories & Computational trajectories $\gamma$ \\
Poincaré recurrence & Solution recognition \\
Molecular collisions & Constraint satisfaction \\
Thermal equilibrium & Categorical completion \\
Entropy $S$ & S-entropy coordinates \\
\end{tabular}
\end{center}
\end{theorem}

This identity is grounded physically: hardware oscillators (CPU cycles, memory access, I/O timing) provide the ``molecular'' substrate. Their timing jitter maps deterministically to S-entropy coordinates through:
\begin{align}
S_k &= \phi_k(\delta_p) \\
S_t &= \phi_t(\delta_p) \\
S_e &= \phi_e(\delta_p)
\end{align}
where $\delta_p = t_{\text{ref}} - t_{\text{local}}$ is the precision-by-difference value.

\subsection{Non-Halting Dynamics and Emergent Memory}

Unlike Turing machines, Poincaré Computing has no halting condition:

\begin{theorem}[Non-Halting Dynamics]
Poincaré Computing systems exhibit:
\begin{enumerate}
    \item \textbf{Continuous exploration}: The system never halts; dynamics continue indefinitely
    \item \textbf{Emergent memory}: The trajectory history constitutes memory without explicit storage
    \item \textbf{Irreversible accumulation}: Computational capability increases monotonically with time
    \item \textbf{Conditional complexity reduction}: Prior exploration reduces complexity for related problems
\end{enumerate}
\end{theorem}

This matches our Infinite Recursion Theorem: the system never reaches ``maximum entropy'' because new categories self-generate through exploration. Time never stops because there is always more phase space to explore.

\subsection{The $\epsilon$-Boundary and Categorical Irreversibility}

Solutions are recognised not at the exact initial state, but at the $\epsilon$-boundary:

\begin{theorem}[$\epsilon$-Boundary Recognition]
Solutions in Poincaré Computing are recognised exactly one categorical step from closure:
\begin{equation}
\text{Solution} \Leftrightarrow \|\gamma(T) - \mathbf{S}_0\| = \epsilon_{\min}
\end{equation}
where $\epsilon_{\min}$ is the minimum categorical distance.
\end{theorem}

\begin{proof}
Categorical irreversibility forbids exact return to the initial state: once a category has been completed, it cannot be ``un-completed.'' The trajectory can approach $\mathbf{S}_0$ arbitrarily closely but cannot coincide with it. Being one step away IS the solution, not an approximation to it.
\end{proof}

This connects to the Gödelian residue: the $\epsilon$-boundary is the computational manifestation of the inaccessible portion $x$ in $\infty - x$. Complete return is structurally impossible, just as complete knowledge of $\mathcal{R}$ is structurally impossible.

\subsection{Identity Unification: No von Neumann Separation}

\begin{theorem}[Identity Unification]
In Poincaré Computing, a categorical state $\mathbf{S} \in \mathcal{S}$ simultaneously encodes:
\begin{enumerate}
    \item \textbf{Memory address}: Position in the $3^k$ hierarchical storage
    \item \textbf{Processor state}: Current configuration of the computation
    \item \textbf{Semantic content}: Meaning through harmonic coincidence
\end{enumerate}
These are projections of the same mathematical object, not transformations between different objects.
\end{theorem}

This eliminates the von Neumann bottleneck at the architectural level. The ``address'' and ``computation'' are the same thing viewed from different perspectives---exactly as the triple equivalence predicts.

\subsection{Answer Equivalence vs. Algorithmic Equivalence}

Poincaré Computing is categorically incomparable with Turing computation:

\begin{theorem}[Incomparability]
Poincaré Computing and Turing machines are distinct computational paradigms:
\begin{center}
\begin{tabular}{ll}
\textbf{Turing} & \textbf{Poincaré} \\
\hline
Algorithmic equivalence & Answer equivalence \\
Instruction sequences & Trajectory geometry \\
Halting condition & Recurrence condition \\
Explicit memory & Emergent memory \\
$O(n)$ operations & $O(\Pi)$ categorical completions \\
\end{tabular}
\end{center}
\end{theorem}

Two Poincaré computations are equivalent if they produce the same answer, regardless of:
\begin{itemize}
    \item Initial state
    \item Constraint structure
    \item Trajectory geometry
    \item Exploration history
\end{itemize}

This is the computational form of the Decoupling Theorem: the path to the solution (trajectory) and the solution itself (recurrence point) are independent.

\subsection{Complexity in Categorical Completions}

\begin{definition}[Poincaré Complexity]
The Poincaré complexity $\Pi(P)$ of a problem $P$ is the minimum number of categorical completions required to recognise solution closure.
\end{definition}

This measure is:
\begin{enumerate}
    \item \textbf{Clock-independent}: Measured per completion, not per second
    \item \textbf{Initial-state agnostic}: The initial state is inferred, not known
    \item \textbf{Path-independent}: Only the number of completions matters, not their order
\end{enumerate}

For a phase space discretised into $N = 3^k$ cells, the expected recurrence time scales as $O(N) = O(3^k)$, yielding logarithmic complexity in the precision parameter $k$.

\subsection{Integration with S-Navigation}

Poincaré Computing implements S-navigation directly:

\begin{enumerate}
    \item \textbf{Problem specification}: The initial state $\mathbf{S}_0$ and constraints $\mathcal{C}$ define a ``target region'' in S-space
    
    \item \textbf{Navigation}: The trajectory $\gamma$ traverses S-space, guided by categorical dynamics
    
    \item \textbf{Recognition}: Recurrence to the $\epsilon$-neighborhood of $\mathbf{S}_0$ signals solution
    
    \item \textbf{Extraction}: The constraint-satisfying trajectory encodes the answer
\end{enumerate}

The Moon Landing Algorithm is a specific navigation strategy within this framework, optimised for efficiency through fuzzy windows and semantic gravity fields.

\subsection{Summary: Computing as Gas Dynamics}

Poincaré Computing establishes that:

\begin{enumerate}
    \item Computation IS trajectory completion in bounded phase space
    \item The S-entropy space $\mathcal{S} = [0,1]^3$ IS the gas chamber
    \item Hardware oscillations ARE the molecular dynamics
    \item Solutions ARE recurrent trajectories satisfying constraints
    \item The $\epsilon$-boundary IS the Gödelian residue made computational
    \item Identity unification ELIMINATES the von Neumann bottleneck
    \item Answer equivalence REPLACES algorithmic equivalence
\end{enumerate}

This is not a new computational paradigm \textit{inspired by} physics. It is the recognition that computation and physics ARE the same dynamics in bounded phase space, viewed from different perspectives.

