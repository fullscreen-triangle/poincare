\section{The Recognition Problem}

\subsection{Finding versus Recognizing}

Navigation can bring us to a location in S-entropy space, but how do we know when we have arrived at the \textit{correct} location? This question connects to fundamental issues in epistemology \cite{gettier1963} and the logic of scientific discovery \cite{popper1959}. This is the Recognition Problem: distinguishing $g = 9.81$ from $g = 9.82$.

\subsection{Consistency as Recognition Criterion}

The answer lies in consistency. The correct value is the unique fixed point where all navigation paths converge.

\begin{definition}[Consistency Function]
For a candidate value $v$ and a set of constraints $\mathcal{C} = \{c_1, c_2, \ldots, c_n\}$, the consistency function is:
\begin{equation}
\mathcal{K}(v) = \sum_{i=1}^{n} \left| c_i(v) - c_i^{\text{observed}} \right|^2
\end{equation}
where $c_i(v)$ is the prediction of constraint $i$ given value $v$, and $c_i^{\text{observed}}$ is the observed value.
\end{definition}

\begin{theorem}[Recognition Criterion]
The correct value $v^*$ minimizes the consistency function:
\begin{equation}
v^* = \arg\min_v \mathcal{K}(v)
\end{equation}
At $v^*$, all constraints are simultaneously satisfied: $\mathcal{K}(v^*) = 0$.
\end{theorem}

\subsection{Example: Recognizing $g = 9.81$}

Consider the constraints on gravitational acceleration:

\begin{enumerate}
    \item Pendulum period: $T = 2\pi\sqrt{L/g}$
    \item Free fall: $d = \frac{1}{2}gt^2$
    \item Orbital mechanics: $g = GM/R^2$
    \item Weight: $W = mg$
    \item Tidal forces: $\Delta g = 2GMr/R^3$
\end{enumerate}

For $g = 9.81$:
\begin{itemize}
    \item Pendulum periods match observations
    \item Free fall distances match observations
    \item Orbital calculations work
    \item Weights are consistent
    \item Tidal predictions are accurate
\end{itemize}

For $g = 9.82$:
\begin{itemize}
    \item Pendulum periods are slightly off
    \item Free fall predictions accumulate error
    \item Orbital mechanics diverge
    \item Weights miscalculated
    \item Tidal predictions fail
\end{itemize}

The value $g = 9.81$ is recognized by its unique consistency across all constraints.

\subsection{S-Distance Gradient Information}

During navigation, the S-distance provides gradient information---a sense of ``warmer'' or ``colder'':

\begin{definition}[S-Gradient]
The S-gradient at position $\mathbf{r}$ is:
\begin{equation}
\nabla_S d_S = \left( \frac{\partial d_S}{\partial S_k}, \frac{\partial d_S}{\partial S_t}, \frac{\partial d_S}{\partial S_e} \right)
\end{equation}
\end{definition}

Moving in the direction of negative gradient reduces S-distance:
\begin{equation}
\mathbf{r}_{n+1} = \mathbf{r}_n - \eta \nabla_S d_S
\end{equation}

where $\eta$ is the step size. As $d_S \to 0$, the navigator approaches the target.

\subsection{Fixed Points in S-Space}

Correct answers are fixed points: locations where further navigation produces no change.

\begin{definition}[Fixed Point]
A location $\mathbf{r}^*$ is a fixed point if:
\begin{equation}
\nabla_S d_S(\mathbf{r}^*) = \mathbf{0}
\end{equation}
and $d_S(\mathbf{r}^*) = 0$.
\end{definition}

\begin{theorem}[Uniqueness of Fixed Points]
For well-posed problems, the fixed point is unique. Multiple inconsistent values cannot all satisfy $\mathcal{K}(v) = 0$.
\end{theorem}

\subsection{Convergence Dynamics}

Navigation converges to fixed points through iterative consistency-checking:

\begin{algorithm}[H]
\caption{Fixed Point Recognition}
\begin{algorithmic}[1]
\State Initialize candidate $v_0$
\Repeat
    \State Compute consistency $\mathcal{K}(v_n)$
    \State Compute gradient $\nabla_v \mathcal{K}$
    \State Update $v_{n+1} = v_n - \eta \nabla_v \mathcal{K}$
\Until{$\mathcal{K}(v_n) < \epsilon$ (convergence threshold)}
\State \Return $v_n$ as recognized answer
\end{algorithmic}
\end{algorithm}

\subsection{Local Minima and Escape}

A concern is convergence to local minima---values that locally minimize inconsistency but are not globally correct.

\begin{proposition}[Local Minimum Escape]
The infinite recursion structure provides escape mechanisms from local minima. If navigation is stuck at a local minimum at recursion level $n$, shifting to level $n+1$ or $n-1$ reveals structure that enables escape.
\end{proposition}

This is analogous to simulated annealing \cite{kirkpatrick1983}: the multi-scale structure provides perturbation mechanisms that prevent permanent trapping.

\subsection{Recognition Without Understanding}

A crucial feature of the recognition criterion is that it operates without requiring understanding:

\begin{quote}
\textit{The navigator need not know \textbf{why} $g = 9.81$ is correct. It suffices to recognize that at $g = 9.81$, all constraints are satisfied simultaneously.}
\end{quote}

Recognition is pattern-matching: does this value fit all the constraints? Understanding is explanation: why does this value fit? These are independent operations.

\subsection{The Viability Threshold}

Not all problems require exact recognition. Many admit a viability range:

\begin{definition}[Viability Threshold]
A value $v$ is viable if $\mathcal{K}(v) < \mathcal{K}_{\text{max}}$ for some acceptable threshold $\mathcal{K}_{\text{max}}$.
\end{definition}

For practical problems, viability often suffices \cite{simon1955}:
\begin{itemize}
    \item Engineering: $g \approx 10$ m/s$^2$ is often good enough
    \item Navigation: Approximate coordinates reach the destination
    \item Decision-making: ``Roughly correct'' enables action
\end{itemize}

The Moon Landing Algorithm exploits viability: it seeks viable solutions rather than demanding perfect recognition. This enables earlier termination and greater efficiency.

\subsection{Recognition in the $3 \times 3$ Matrix}

Recognition can be performed using any column of the structural matrix:

\begin{itemize}
    \item \textbf{Oscillatory recognition}: Does the value produce consistent frequencies?
    \item \textbf{Categorical recognition}: Does the value produce consistent category counts?
    \item \textbf{Partition recognition}: Does the value produce consistent selectivities?
\end{itemize}

By the triple equivalence, all three produce the same recognition outcome. The choice is one of convenience.

