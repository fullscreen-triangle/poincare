\section{Universal Accessibility Theorem}

\subsection{The Universality Claim}

We now establish the strongest claim of post-explanatory epistemology: any sentient system can access any truth. This universality connects to fundamental limits of computation \cite{turing1936} and the relationship between physical systems and information processing \cite{landauer1961}.

\begin{theorem}[Universal Accessibility]
Let $\Sigma$ be any sentient system embedded in reality, and let $\tau$ be any truth about reality. Then there exists a navigation path $\pi_\Sigma$ such that $\Sigma$ can traverse $\pi_\Sigma$ to reach $\tau$.
\end{theorem}

\subsection{Conditions for Sentience}

For the theorem to apply, a system must satisfy minimal conditions:

\begin{definition}[Sentient System]
A system $\Sigma$ is sentient if:
\begin{enumerate}
    \item \textbf{Embedded}: $\Sigma$ is located within the same reality as the truths it seeks
    \item \textbf{Bounded}: $\Sigma$ satisfies the Bounded System Axiom (finite extent, energy, duration)
    \item \textbf{Processing}: $\Sigma$ can perform finite computations on information
    \item \textbf{Memory}: $\Sigma$ can store and retrieve information across time
\end{enumerate}
\end{definition}

These conditions are minimal. They exclude rocks and thermostats but include humans, cows, aliens, and sufficiently advanced artificial systems.

\subsection{Proof of Universal Accessibility}

\begin{proof}
We prove constructively that any sentient system can reach any truth:

\textbf{Step 1}: By the Bounded System Axiom, both $\Sigma$ and the domain containing truth $\tau$ are bounded systems. Therefore both admit triple equivalence representations.

\textbf{Step 2}: The S-entropy space containing $\tau$ is the same space in which $\Sigma$ is embedded. They share the same $3 \times 3$ structural matrix.

\textbf{Step 3}: By the infinite recursion theorem, there exist infinitely many equivalent expressions for any S-coordinate. Some subset of these expressions is accessible to $\Sigma$'s particular sensory and cognitive apparatus.

\textbf{Step 4}: A path $\pi_\Sigma$ can be constructed using only expressions accessible to $\Sigma$:
\begin{itemize}
    \item If $\Sigma$ perceives oscillations, use oscillatory expressions
    \item If $\Sigma$ counts categories, use categorical expressions
    \item If $\Sigma$ detects selectivity, use partition expressions
    \item If $\Sigma$ has none of these, use combinations at different recursion levels
\end{itemize}

\textbf{Step 5}: By the path independence theorem, $\pi_\Sigma$ reaches the same truth $\tau$ as any other path. The destination is invariant under path choice.

Therefore $\Sigma$ can reach $\tau$.
\end{proof}

\subsection{Species-Independent Science}

The theorem establishes that science is truly universal:

\begin{corollary}[Species-Independent Science]
The truths of physics, mathematics, and logic are accessible to any species capable of sentience. The methods may differ; the truths are identical.
\end{corollary}

This resolves the apparent tension between universal truth and particular method:
\begin{itemize}
    \item Humans use human apparatus (eyes, rulers, pendulums)
    \item Cows would use cow apparatus (body sensation, grazing experience)
    \item Aliens use alien apparatus (whatever senses and tools they have)
    \item All arrive at $g = 9.81$ in their respective units
\end{itemize}

\subsection{The Accessibility Spectrum}

While all truths are accessible, accessibility varies:

\begin{definition}[Accessibility Distance]
For sentient system $\Sigma$ and truth $\tau$, the accessibility distance is:
\begin{equation}
d_{\text{access}}(\Sigma, \tau) = \min_{\pi \in \Pi_\Sigma} \text{length}(\pi)
\end{equation}
where $\Pi_\Sigma$ is the set of paths traversable by $\Sigma$.
\end{definition}

Some truths are ``close'' to a given sentient system (easily accessible with minimal navigation), while others are ``far'' (requiring long navigation paths).

\begin{example}
For humans:
\begin{itemize}
    \item ``Fire is hot'' is close: direct sensory access
    \item ``$g = 9.81$'' is medium: requires simple apparatus
    \item ``Higgs boson mass = 125 GeV'' is far: requires extensive technological mediation
\end{itemize}

For an alien with different senses, the distances might be entirely different.
\end{example}

\subsection{Cognitive Diversity as Path Diversity}

Different cognitive architectures correspond to different preferred paths through S-space:

\begin{itemize}
    \item \textbf{Visual thinkers}: Navigate via oscillatory patterns (waveforms, frequencies)
    \item \textbf{Logical thinkers}: Navigate via categorical structures (sets, relations)
    \item \textbf{Intuitive thinkers}: Navigate via partition patterns (selection, flow)
\end{itemize}

No cognitive style is privileged. All reach the same truths through their preferred paths.

\subsection{Machine Intelligence and Accessibility}

Artificial intelligence systems are sentient in the required sense:
\begin{itemize}
    \item Embedded in reality (hardware exists in spacetime)
    \item Bounded (finite memory, energy, processing time)
    \item Processing-capable (computation is their core function)
    \item Memory-equipped (storage is fundamental to computing)
\end{itemize}

\begin{corollary}[AI Accessibility]
Artificial intelligence systems can access all truths accessible to biological sentient systems. There is no truth that humans can reach but AI cannot, and vice versa.
\end{corollary}

The practical difference is accessibility distance: some truths are closer to human navigation styles, others to machine navigation styles.

\subsection{Communication Across Sentient Systems}

Universal accessibility implies that communication of truths across sentient systems is possible:

\begin{proposition}[Cross-System Communication]
If $\Sigma_1$ and $\Sigma_2$ are both sentient systems that have reached truth $\tau$, they can verify agreement despite using different paths and representations.
\end{proposition}

\begin{proof}
Both systems are at the same location in S-space (truth $\tau$). They can compare predictions derived from $\tau$ and verify consistency. The verification does not require that they used the same path or representation.
\end{proof}

This is how a human scientist and an alien scientist could verify they have discovered the same physical law, despite radically different apparatus and cognitive architectures.

\subsection{Limits of Accessibility}

Universal accessibility has limits:

\begin{enumerate}
    \item \textbf{Time}: Some truths require navigation paths longer than the system's lifespan
    \item \textbf{Computation}: Some truths require more computation than the system can perform
    \item \textbf{Energy}: Some truths require more energy than is available
\end{enumerate}

These are practical limits, not principled ones. Given sufficient time, computation, and energy, any truth is accessible.

\begin{definition}[Practical Accessibility]
Truth $\tau$ is practically accessible to $\Sigma$ if:
\begin{equation}
d_{\text{access}}(\Sigma, \tau) < \min(T_\Sigma, C_\Sigma, E_\Sigma)
\end{equation}
where $T_\Sigma$, $C_\Sigma$, $E_\Sigma$ are the system's time, computational, and energy budgets.
\end{definition}

\subsection{The Democratic Nature of Truth}

The Universal Accessibility Theorem establishes that truth is fundamentally democratic:

\begin{quote}
\textit{No sentient system has privileged access to truth. All are embedded in the same S-entropy space. All can navigate to the same locations. Truth belongs to everyone who can reach it.}
\end{quote}

This is the deepest implication of post-explanatory epistemology: knowledge is not the possession of a privileged few but a location in a shared space, accessible to any traveler capable of the journey.

