\section{Triple Equivalence Foundation}

\subsection{The Bounded System Premise}

We begin with a single foundational premise: all physical systems encountered in reality are bounded. This constraint has fundamental implications for information processing \cite{bekenstein1981} and computation \cite{lloyd2000}.

\begin{axiom}[Bounded System Axiom]
Every physical system $\Sigma$ satisfies three finiteness conditions:
\begin{enumerate}
    \item \textbf{Spatial boundedness}: There exists $L < \infty$ such that $\Sigma$ is contained within a region of diameter $L$
    \item \textbf{Energetic boundedness}: There exists $E_{\max} < \infty$ such that the total energy of $\Sigma$ satisfies $E \leq E_{\max}$
    \item \textbf{Temporal boundedness}: There exists $T < \infty$ such that any process in $\Sigma$ completes within time $T$
\end{enumerate}
\end{axiom}

This axiom excludes only mathematical idealizations: infinite planes, unbounded potentials, eternal processes. Every system we can observe, measure, or interact with is bounded in all three senses.

\subsection{Three Perspectives on Bounded Systems}

A bounded system can be described from three distinct perspectives, each arising naturally from the boundedness conditions.

\subsubsection{The Oscillatory Perspective}

Spatial and energetic boundedness imply that motion within the system must reverse. A particle moving in a bounded region must eventually turn back; energy conservation in a bounded potential requires periodic behavior. This generates oscillation.

\begin{definition}[Oscillatory Description]
The oscillatory description of a bounded system consists of:
\begin{itemize}
    \item Frequency spectrum $\{\omega_i\}$ of fundamental modes
    \item Amplitude set $\{A_i\}$ for each mode
    \item Phase configuration $\{\phi_i\}$ at any instant
    \item Period $T = 2\pi/\omega_{\min}$ of the fundamental oscillation
\end{itemize}
\end{definition}

The system's state at any time is fully specified by $(A_i, \phi_i)$ for all modes. This perspective connects to classical mechanics \cite{arnold1989} and dynamical systems theory \cite{strogatz1994}.

\subsubsection{The Categorical Perspective}

Temporal boundedness implies that continuous motion is partitioned into distinguishable states. If a process completes in finite time, it passes through a finite sequence of configurations. These configurations are categories.

\begin{definition}[Categorical Description]
The categorical description of a bounded system consists of:
\begin{itemize}
    \item Category count $M$ = number of distinguishable states
    \item Category depth $n$ = number of elements per category
    \item Transition graph $G = (V, E)$ where vertices are categories and edges are allowed transitions
    \item Actualization sequence $\{c_1, c_2, \ldots, c_k\}$ of categories traversed
\end{itemize}
\end{definition}

The system's evolution is fully specified by which categories it occupies at each moment. This categorical viewpoint has formal connections to category theory \cite{maclane1971, awodey2010}.

\subsubsection{The Partition Perspective}

The combined boundedness conditions imply that any observation of the system involves selecting among finite alternatives. Measurement partitions the state space into distinguishable outcomes.

\begin{definition}[Partition Description]
The partition description of a bounded system consists of:
\begin{itemize}
    \item Aperture set $\{a_1, a_2, \ldots, a_k\}$ through which the system can pass
    \item Selectivity $s_a$ of each aperture (probability of passage)
    \item Partition lag $\tau_a$ = time to traverse aperture $a$
    \item Aperture potential $\Phi_a = -k_B T \ln s_a$
\end{itemize}
\end{definition}

The system's dynamics are fully specified by which apertures it traverses and with what selectivity. The partition perspective connects to combinatorial analysis \cite{andrews1976, hardy1918} and statistical mechanics \cite{gibbs1902}.

\subsection{The Triple Equivalence Theorem}

The three descriptions are not merely compatible; they are mathematically identical.

\begin{theorem}[Triple Equivalence]
For any bounded system $\Sigma$, the oscillatory, categorical, and partition descriptions are equivalent in the following precise sense:
\begin{enumerate}
    \item Period = Category traversal time = Sum of partition lags:
    \begin{equation}
    T = M \cdot \langle\tau_p\rangle = \sum_a \tau_a
    \end{equation}
    
    \item Amplitude ratio = Category depth = Inverse selectivity:
    \begin{equation}
    \frac{A}{A_0} = n = \frac{1}{s}
    \end{equation}
    
    \item Phase space entropy = Categorical entropy = Partition entropy:
    \begin{equation}
    S = k_B \sum_i \ln\left(\frac{A_i}{A_0}\right) = k_B M \ln n = k_B \sum_a \ln\left(\frac{1}{s_a}\right)
    \end{equation}
\end{enumerate}
\end{theorem}

\begin{proof}
We prove each equivalence:

\textbf{(1) Temporal equivalence:} Consider a system completing one oscillation period $T$. During this period, it transitions through $M$ categorical states, spending average time $\langle\tau_p\rangle$ in each. Thus $T = M \cdot \langle\tau_p\rangle$. Each categorical transition involves passage through an aperture with lag $\tau_a$, so $T = \sum_a \tau_a$.

\textbf{(2) Amplitude-depth-selectivity equivalence:} The oscillation amplitude $A$ relative to ground state $A_0$ determines the number of distinguishable energy levels accessible, which equals the categorical depth $n$. An aperture with selectivity $s = 1/n$ permits exactly $n$ outcomes, so $A/A_0 = n = 1/s$.

\textbf{(3) Entropy equivalence:} The phase space volume accessible to an oscillator with amplitude $A$ scales as $(A/A_0)^2$, giving entropy $S = k_B \sum_i \ln(A_i/A_0)$. The number of categorical configurations is $n^M$, giving entropy $S = k_B M \ln n$. The partition entropy sums over aperture contributions: $S = k_B \sum_a \ln(1/s_a) = k_B \sum_a \ln n_a$. All three reduce to the same expression.
\end{proof}

\subsection{Implications of Triple Equivalence}

The theorem has profound implications:

\begin{corollary}[Representational Freedom]
Any calculation performed in one representation yields identical results when translated to the other representations. There is no privileged description.
\end{corollary}

\begin{corollary}[Multiple Valid Explanations]
Any phenomenon admits three equally valid explanations:
\begin{itemize}
    \item Oscillatory: ``It happens because frequencies resonate''
    \item Categorical: ``It happens because categories fill''
    \item Partition: ``It happens because apertures select''
\end{itemize}
None is more fundamental than the others.
\end{corollary}

\begin{corollary}[Cross-Domain Transfer]
Solutions discovered in one domain (e.g., acoustics) transfer to other domains (e.g., quantum mechanics) because both are bounded systems obeying the same triple equivalence.
\end{corollary}

The triple equivalence is not an approximation or a useful fiction. It is a mathematical identity that holds exactly for all bounded systems.

